\documentclass[10pt,a4paper]{article}
\usepackage[utf8]{luainputenc}
\usepackage[T1]{fontenc}
\usepackage[margin=0.75in]{geometry}
\usepackage{enumitem}
\usepackage{titlesec}
\usepackage{xcolor}
\usepackage{hyperref}
\usepackage{array}
\usepackage{tabularx}
\usepackage{longtable}
\usepackage{multicol}

% Colors
\definecolor{darkblue}{RGB}{0,51,102}
\definecolor{mediumblue}{RGB}{0,102,204}
\definecolor{lightgray}{RGB}{240,240,240}

% Hyperlink setup
\hypersetup{
    colorlinks=true,
    linkcolor=darkblue,
    urlcolor=mediumblue,
    citecolor=darkblue
}

% Section formatting
\titleformat{\section}{\large\bfseries\color{darkblue}}{}{0em}{}[\titlerule]
\titleformat{\subsection}{\normalsize\bfseries\color{mediumblue}}{}{0em}{}
\titleformat{\subsubsection}{\small\bfseries}{}{0em}{}

% List formatting
\setlist[itemize]{leftmargin=1em,topsep=0.5em,itemsep=0.25em}
\setlist[enumerate]{leftmargin=1em,topsep=0.5em,itemsep=0.25em}

% Custom commands
\newcommand{\role}[4]{\textbf{#1} \hfill \textit{#2} \\ \textit{#3} \hfill #4}
\newcommand{\highlight}[1]{\textcolor{mediumblue}{\textbf{#1}}}

% Reduce spacing
\setlength{\parskip}{0.5em}
\setlength{\parindent}{0pt}

\begin{document}

% Header
\begin{center}
    {\huge\textbf{\color{darkblue}Markus Schulte}} \\[0.5em]
    {\Large\textbf{Anti-CV: A Refreshingly Honest Career Retrospective}} \\[0.5em]
    \textit{``The failures that shaped me, the lessons I learned the hard way, and why I'm still employable''} \\[0.5em]
    \small Cloud Architect \& Software Engineer $\bullet$ Cologne, Germany \\
    \href{mailto:mail@schulte-development.de}{mail@schulte-development.de} $\bullet$ +49 178 7217768 $\bullet$ \href{https://schulte-development.de}{schulte-development.de}
\end{center}

\section{The Uncomfortable Truth About My Career}

After 18+ years in software development, I've learned that the most valuable lessons come from spectacular failures, questionable decisions, and the occasional stroke of accidental genius. This Anti-CV celebrates the beautiful messiness of a real career -- one where I've been promoted beyond my competence, made technology choices that aged like milk, and somehow still managed to deliver value.

\textbf{What I Actually Bring to the Table:}
\begin{itemize}
    \item \highlight{Hard-Won Wisdom:} 18+ years of mistakes that taught me more than any certification
    \item \highlight{Honest Leadership:} I've failed at management, learned from it, and become better
    \item \highlight{Technology Pragmatism:} Survived enough hype cycles to spot genuine value
    \item \highlight{Resilient Architecture:} Built systems that survive despite my imperfect decisions
\end{itemize}

\subsection{Career Highlights (The Good, Bad, and Embarrassing)}

\begin{tabularx}{\textwidth}{|l|X|}
\hline
\textbf{Epic Wins} & Reduced deployment time from days to 30 minutes; Built platforms serving 9M+ users; Saved 60 person-days with one decisive architecture decision \\
\hline
\textbf{Spectacular Fails} & Promoted to Head of Development without management experience; Chose Sphinx over ElasticSearch (oops); Underestimated the human factor in leadership \\
\hline
\textbf{Lucky Breaks} & Started freelancing just as cloud computing exploded; Discovered Golang when microservices were trendy; Joined companies right before major transformations \\
\hline
\textbf{Honest Expertise} & Java (18y of love-hate), Cloud Architecture (AWS/Azure/GCP), Golang (my current favorite), Microservices (with healthy skepticism) \\
\hline
\end{tabularx}

\section{Recent Adventures in Enterprise Chaos}

\subsection{November 2024 -- Present: Cloud Architect at LBBW}
\textit{``Phoenix'' IT Modernization Project -- Because every bank needs a rebirth}

\textbf{The Challenge:} Help migrate a major German state bank's legacy applications to Azure's ``Skywalker'' platform. Yes, they really named it after Star Wars characters. No, I didn't choose the names.

\textbf{Reality Check:} Cataloging hundreds of legacy applications is like archaeological work -- you never know what ancient horrors you'll uncover. Currently applying R-methodology to assess cloud migration readiness while the platform itself is still evolving.

\textbf{Key Contributions:}
\begin{itemize}
    \item Architecture governance for 10 of 30 applications (the lucky ones)
    \item Migration strategy development using proven Azure patterns
    \item Technology evaluation including Elastisys Welkin (vendor flexibility is key)
    \item Cost analysis that made executives wince but proceed anyway
\end{itemize}

\subsection{October 2021 -- December 2024: Cloud Architect at Union Investment}
\textit{The Great InvestmentWelt Modernization -- 3+ Years of Beautiful Complexity}

\textbf{The Mission:} Transform a legacy B2B sales platform into a modern, distributed architecture that could serve as a template for the entire corporate group. Spoiler alert: it worked.

\textbf{The Plot Twist:} What started as a simple modernization became a masterclass in Self-contained Systems, microfrontends, and the art of making autonomous teams actually work together.

\textbf{My Favorite Failures \& Recoveries:}
\begin{itemize}
    \item \textbf{Team Building Reality:} Had to make difficult personnel decisions -- turns out not everyone fits despite best intentions
    \item \textbf{Architecture Simplicity:} Sometimes the obvious solution (Azure Event Hub + Algolia) is actually the right one
    \item \textbf{Bias for Action:} Saved 60 person-days by implementing Web Components immediately after PoC instead of endless analysis
\end{itemize}

\textbf{Actual Achievements:}
\begin{itemize}
    \item Built microfrontend toolkit enabling autonomous team development
    \item Designed event-driven search architecture across 8 domain boundaries
    \item Achieved 99.9\% uptime while serving 80,000+ users
    \item Reduced deployment time from days to 30 minutes
    \item Led teams of 5-7 developers as technical Product Owner
\end{itemize}

\textbf{360° Feedback Highlights:} ``Brings exactly what we need'' and ``Makes work enjoyable and professional.'' Also: ``Consider pace of technical explanations'' -- apparently I get excited about architecture.

\subsection{December 2021 -- November 2024: Middleware Engineer at Saloodo!}
\textit{Part-Time Golang Adventure -- My First Professional Go Rodeo}

\textbf{The Setup:} DHL's digital freight platform needed middleware to sync ~10 million records with Salesforce. I took this part-time gig specifically to use Golang professionally for the first time.

\textbf{The Reality:} Sole developer responsible for mission-critical data synchronization. No pressure, right?

\textbf{What I Learned:}
\begin{itemize}
    \item Quality standards pay for themselves even (especially!) when you're working alone
    \item Saloodo's database had more inconsistencies than a time travel movie
    \item 100\% uptime is achievable with proper error handling and testing
    \item The middleware was architecturally unnecessary -- AWS Lambda would have been simpler
\end{itemize}

\textbf{Technical Win:} Achieved 70\% test coverage on 2,776 lines of code with 5-second test execution using Dockertest.

\section{The Learning Years (Where I Made My Best Mistakes)}

\subsection{June 2018 -- October 2019: Cloud Consultant at Trusted Shops}
\textit{My First Real Microservices Experience -- Spoiler: It Was Complicated}

\textbf{The Dream:} Work with a modern microservices architecture on AWS, implement contract testing, modernize legacy systems.

\textbf{The Reality:} Microservices are hard. Really hard. Simple changes required modifications across many services, and without proper testing and automation, it became a maintenance nightmare.

\textbf{The Lesson:} This experience made me permanently skeptical of microservices as an overarching architecture. I now prefer Self-contained Systems -- they give you the benefits without the operational complexity.

\textbf{What I Actually Accomplished:}
\begin{itemize}
    \item Modernized JavaEE6 monolith to cloud-native architecture
    \item Implemented Salesforce/Zuora integration for subscription management
    \item Led Pact contract testing training across multiple teams
    \item Built SMS integration with Twilio (because everything needs SMS alerts)
\end{itemize}

\subsection{October 2016 -- September 2017: Senior Engineer at toom Baumarkt}
\textit{The ``Butterkekse'' Team -- Yes, We Were Really Called Butter Biscuits}

\textbf{The Good:} First encounter with Docker and Kubernetes, quality leadership across 4 teams, ``you build it, you run it'' philosophy.

\textbf{The Bad:} Kubernetes was (and still is) overly complex for most applications. It's a tool for IT operations providers, not simple business applications.

\textbf{The Ugly:} Achieved 70\% improvement in test execution speed through optimization -- because waiting for tests is the worst part of development.

\subsection{August 2007 -- June 2014: The wer-kennt-wen Saga}
\textit{From Developer to Head of Development to Developer Again -- A Masterclass in Career Humility}

\textbf{The Platform:} One of Germany's largest social networks with 9+ million users. Think Facebook, but German and before Facebook dominated.

\textbf{My Epic Management Failure:} Accepted Head of Development role in 2008 without management experience. Both leadership and I assumed technical competence would translate to leadership effectiveness. Spoiler: it doesn't.

\textbf{What I Learned the Hard Way:}
\begin{itemize}
    \item Technical skills $\neq$ Management skills
    \item The human factor is everything in team leadership
    \item Laissez-faire leadership suited my personality better than micromanagement
    \item Sometimes stepping back is the best way forward
\end{itemize}

\textbf{Technology Fail:} Chose Sphinx over ElasticSearch because it seemed familiar. ElasticSearch was the right choice long-term, and we eventually migrated. Conservative technology choices aren't always safe choices.

\textbf{Actual Achievements:}
\begin{itemize}
    \item Scaled architecture for 9+ million users
    \item Migrated from procedural to object-oriented architecture
    \item Implemented comprehensive security measures
    \item Introduced CI/CD with Jenkins
    \item Reduced page load times by 60\%
\end{itemize}

\section{Personal Projects (Where I Experiment Without Consequences)}

\subsection{2025: SchulteDev Portfolio Repository}
AI-powered CV generation using Claude 4 because apparently I needed automation to write about my failures. Features GitHub Actions workflows that are more complex than some production systems I've worked on.

\subsection{2025: AI Document Processing for Tax Software}
Built a CLI tool using Azure Document Intelligence to automate tax document processing. Because nothing says ``I'm a cloud architect'' like automating your own taxes.

\subsection{2018: Jenkins ↔ GitHub Integration PoC}
Attempted to build automatic Jenkins setup for GitHub repositories. The project worked but was solving a problem that GitHub Actions later solved better.

\subsection{2015: Crypto-Arbitrage Trading Bot}
Created Bitcoin arbitrage algorithms with machine learning. Learned that the crypto market is even more irrational than software estimates.

\section{Education \& The Long Road to Wisdom}

\subsection{2004 -- 2012: Diploma in Computer Science, University of Koblenz-Landau}
\textit{The Degree That Took Forever -- But For Good Reasons}

\textbf{Degree:} Diploma in Computer Science (equivalent to Master's degree)
\textbf{Grade:} Distinction (magna cum laude equivalent)
\textbf{Duration:} 8 years (standard: 5 years)

\textbf{Why It Took So Long:} Started working full-time at wer-kennt-wen in 2007 while still studying. Prioritized real-world experience over academic speed. No regrets -- the practical experience was invaluable.

\textbf{Focus Areas:} Computational Visualistics, Software Engineering, Human-Computer Interaction

\subsection{2003: Abitur (German High School Diploma)}
\textbf{School:} Gymnasium Haren (Ems), Germany
\textbf{Advanced Courses:} Mathematics and Physics

\textbf{First IT Contact:} Taught myself PHP in 2000-2003 using a PHP reference book. Created forums for friend groups and graduating class. Discovered that I enjoyed building things that people actually used.

\subsection{Key Certifications \& Recognition}
\begin{itemize}
    \item \textbf{StackOverflow:} 4000+ reputation points from helping others solve problems
    \item \textbf{Java SE 8 TechCheck:} 93/100 score, 88th percentile (2018)
    \item \textbf{GitHub:} Open-source contributor with questionable commit messages
\end{itemize}

\section{Technology Stack (What I Actually Know vs. What I Claim)}

\begin{tabularx}{\textwidth}{|l|X|}
\hline
\textbf{Languages I Love} & Java (18y of Stockholm syndrome), Golang (4y of genuine enthusiasm), TypeScript (when forced) \\
\hline
\textbf{Languages I Survive} & PHP (8y of legacy maintenance), JavaScript (inevitable), Shell scripting (dark magic) \\
\hline
\textbf{Cloud Platforms} & AWS (7y), Azure (4y), GCP (occasional consulting) -- I'm platform agnostic, which means I complain about all of them equally \\
\hline
\textbf{Frameworks I Trust} & Spring Boot (reliable), Micronaut (lightweight), Quarkus (fast startup) \\
\hline
\textbf{Tools I Depend On} & Docker (containerization salvation), Terraform (infrastructure as code), Jenkins/GitHub Actions (CI/CD) \\
\hline
\textbf{Databases} & PostgreSQL (preferred), MySQL (nostalgic), Oracle (enterprise reality), Redis (caching sanity) \\
\hline
\textbf{Architecture Patterns} & Self-contained Systems (learned from microservices pain), Event-driven architecture, Clean Architecture principles \\
\hline
\end{tabularx}

\section{Career Breaks (The Honest Gaps)}

\subsection{April 2020 -- September 2021: Parental Leave}
Combined with COVID-19 pandemic impact on freelance market. Spent quality time with family while the industry figured out remote work.

\subsection{August 2015 -- December 2015: Personal Travel Break}
Extended travel to explore different countries and cultures. Sometimes you need to step away from screens to gain perspective.

\subsection{August 2003 -- March 2004: Military Service}
Mandatory German military service in the air force. Learned that military precision and software development timelines are incompatible concepts.

\section{What This All Means}

After 18+ years in this industry, I've learned that the best developers aren't the ones who never make mistakes -- they're the ones who make mistakes quickly, learn from them, and share the lessons with others.

I've built systems that serve millions of users, led teams through complex transformations, and made technology choices that seemed smart at the time. Some worked brilliantly, others taught me valuable lessons about humility.

What I offer now is not perfection, but wisdom gained through experience, failure, and recovery. I know how to build robust systems, lead technical teams, and make architecture decisions that balance pragmatism with innovation.

Most importantly, I've learned that technology is just a tool -- the real value comes from understanding business needs, building effective teams, and creating solutions that actually work in the real world.

\vfill

\begin{center}
\footnotesize
\textit{This Anti-CV was generated with assistance from Claude 4 (Anthropic) based on comprehensive professional timeline data.}\\
\textit{All career information, achievements, and failures are factually accurate -- I wouldn't dare make this stuff up.}\\
\textit{For a traditional CV or specific role requirements, please contact me directly.}
\end{center}

\end{document}