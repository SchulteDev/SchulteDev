\documentclass[10pt,a4paper]{article}
\usepackage[utf8]{luainputenc}
\usepackage[T1]{fontenc}
\usepackage[margin=1.5cm]{geometry}
\usepackage{enumitem}
\usepackage{titlesec}
\usepackage{xcolor}
\usepackage{hyperref}
\usepackage{tabularx}
\usepackage{multicol}

\definecolor{darkgray}{gray}{0.3}
\definecolor{medgray}{gray}{0.5}

\hypersetup{
    colorlinks=true,
    linkcolor=blue,
    urlcolor=blue,
    pdftitle={Anti-CV: Markus Schulte},
    pdfauthor={Markus Schulte}
}

\titleformat{\section}{\Large\bfseries\color{darkgray}}{}{0em}{}[\titlerule]
\titleformat{\subsection}{\large\bfseries\color{medgray}}{}{0em}{}
\titlespacing*{\section}{0pt}{8pt}{4pt}
\titlespacing*{\subsection}{0pt}{6pt}{2pt}

\setlist[itemize]{leftmargin=*,topsep=2pt,itemsep=1pt,parsep=0pt}
\setlength{\parindent}{0pt}
\setlength{\parskip}{4pt}
\pagestyle{empty}

\begin{document}

\begin{center}
{\Huge\bfseries Markus Schulte}\\[4pt]
{\Large\color{medgray} The Anti-CV: What I Learned from Getting It Wrong}\\[6pt]
\small Cloud Architect \& Software Engineer $\cdot$ 19+ Years of Beautiful Mistakes\\[2pt]
Cologne, Germany $\cdot$ \href{mailto:mail@schulte-development.de}{mail@schulte-development.de} $\cdot$ +49 178 7217768\\
\href{https://schulte-development.de}{schulte-development.de} $\cdot$ 
\href{https://linkedin.com/in/markus-schulte}{LinkedIn} $\cdot$ 
\href{https://github.com/SchulteDev}{GitHub}
\end{center}

\vspace{-4pt}

\section*{Professional Summary: The Honest Version}

After 19 years of breaking things in production, choosing the wrong technologies, and occasionally getting it right by accident, I've developed a healthy skepticism toward buzzwords and an unhealthy addiction to over-engineering. Started as a PHP developer when ``<?php'' was considered elegant, survived the Kubernetes hype without drinking the Kool-Aid, and learned that sometimes the simplest solution is hidden behind three microservices and an event bus you didn't need.

\textbf{What I Actually Learned:}
\begin{itemize}
\item Technical excellence means knowing when \textit{not} to use the shiny new framework
\item Leadership is 10\% architecture decisions, 90\% admitting when your architecture was wrong
\item ``You build it, you run it'' is great until you're the only one running it at 3am
\item Every ``temporary solution'' becomes permanent---plan accordingly
\item Kubernetes is amazing if you enjoy spending more time on infrastructure than features
\end{itemize}

\vspace{-2pt}

\section*{Key Impacts: The Numbers Behind the Chaos}

\begin{itemize}
\item \textbf{Platform Scale:} Delivered systems serving 9M+ users (learned that scale reveals every architectural mistake exponentially)
\item \textbf{Deployment Speed:} Reduced deployment time from days to 30 minutes (because fixing production faster is better than not breaking it)
\item \textbf{Team Leadership:} Managed 5--9 developers (discovered management is harder than distributed systems)
\item \textbf{Production Uptime:} 99.9\% uptime at Union Investment, 100\% at Saloodo! (turns out comprehensive testing actually works)
\item \textbf{Cost Optimization:} One well-timed decision saved 60 person-days (bias for action beats analysis paralysis)
\item \textbf{Test Coverage:} Built systems with 1000+ tests running in 10 minutes (because debugging is twice as hard as writing code)
\end{itemize}

\vspace{2pt}

\section*{Technical Expertise: Technologies I've Loved and Lost}

\begin{center}
\renewcommand{\arraystretch}{1.1}
\begin{tabularx}{\textwidth}{lX}
\textbf{Languages} & Java (19y of Stockholm Syndrome), Golang (5y of ``why didn't I learn this sooner''), PHP (8y I'd rather forget), TypeScript (when forced)\\[2pt]
\textbf{Cloud Platforms} & Azure (current obsession), AWS (former obsession), GCP (occasional flirtation), Kubernetes (respectful avoidance)\\[2pt]
\textbf{Architecture} & Microservices (skeptic), Self-contained Systems (convert), Event-driven (survivor), Monolith (nostalgic)\\[2pt]
\textbf{Databases} & PostgreSQL, MySQL, Oracle, MS-SQL, Redis---if it stores data, I've misconfigured it\\[2pt]
\textbf{Testing} & JUnit, Mockito, Pact, Selenium, Dockertest---learned the hard way that untested code is just optimistic fiction\\[2pt]
\textbf{DevOps} & Terraform, Docker, Jenkins, GitHub Actions---because manual deployment builds character (and incidents)\\[2pt]
\textbf{Mistakes} & Sphinx over ElasticSearch, accepting Head of Dev too early, unnecessary middleware, trusting ``temporary'' solutions\\
\end{tabularx}
\end{center}

\vspace{-2pt}

\textit{``Crafting solutions that thrive today and tomorrow''---unless I chose the wrong database, then we refactor tomorrow.}

\vfill

\newpage

\section*{Recent Professional Experience: The Learning Curve}

\subsection*{Cloud Architect at LBBW (Nov 2024 -- Oct 2025)}
\textit{Phoenix IT Modernization $\cdot$ Because every bank needs to migrate hundreds of legacy apps to Azure}

The current adventure: cataloging a delightful portfolio of applications built when SOAP was cutting-edge, convincing teams that ``it works on my machine'' isn't an Azure migration strategy, and explaining why the Skywalker platform needs actual workloads before we can call it ``production-ready.''

\textbf{Beautiful Disasters and Lessons:}
\begin{itemize}
\item \textbf{Platform Gap Analysis:} Discovered that a production-unused platform has... gaps. Shocking, I know. Led enhancement initiatives to make Skywalker actually useful for real applications beyond hello-world demos.
\item \textbf{Legacy Assessment:} Applied R-methodology to hundreds of applications, learning that ``legacy'' is a spectrum from ``slightly dated'' to ``run by a team that retired in 2003.'' Created evaluation frameworks when ``rehost, refactor, or retire'' proved insufficient.
\item \textbf{Enterprise Compliance Theater:} Navigated banking regulations proving that security questionnaires and architecture reviews are the real enterprise product. The software is just an excuse for documentation.
\item \textbf{Cost Reality Check:} Used Azure Calculator to show that cloud migration isn't automatically cheaper---it's just expensive in different, more transparent ways.
\end{itemize}

\vspace{4pt}

\subsection*{Cloud Architect at Union Investment (Oct 2021 -- Dec 2024)}
\textit{3-Year Transformation $\cdot$ From Liferay Monolith to Self-contained Systems Paradise}

Led modernization of InvestmentWelt B2B platform serving 80,000 users who absolutely noticed when we broke things. Spent three years building the microfrontend toolkit that would become the corporate template, proving that good architecture scales better than microservices mythology.

\textbf{Hard-Won Victories and Face-Plants:}
\begin{itemize}
\item \textbf{The Web Components Gamble:} After one PoC, skipped comparison analysis and went straight to implementation. This ``bias for action'' saved 60 person-days. Sometimes the best decision is making one quickly. (Don't tell the architecture review board.)
\item \textbf{Algolia vs ElasticSearch:} Did the deep technical analysis this time, learned from Sphinx mistake. Chose Algolia for simplified ops and richer features. When you operate it yourself, simple wins.
\item \textbf{Event-Driven Enlightenment:} Designed search consuming event streams from eight domains. Turns out Pub/Sub is the answer when ``how do we integrate?'' seems impossible. Who knew message queues from the 1980s would save modern architecture?
\item \textbf{Team Composition Reality:} Built team from scratch, made hard calls to remove members who didn't fit. Learned that team composition matters more than technical brilliance. Removing the wrong people is leadership, not failure.
\item \textbf{Production Excellence:} Achieved 99.9\% uptime and 30-minute emergency deployments. Quality standards and comprehensive testing actually work. Revolutionary concept, apparently.
\item \textbf{360° Feedback Growth:} Consistently rated 9-10/10 technically, but learned to slow down technical explanations for business stakeholders. Being right faster than people can follow is just being unclear.
\end{itemize}

\vspace{4pt}

\subsection*{Middleware Engineer at Saloodo!/DHL (Dec 2021 -- Nov 2024)}
\textit{Part-Time $\cdot$ First Production Golang $\cdot$ Proof That Simple Architecture Works}

Built middleware synchronizing 10M records between Saloodo and Salesforce. Achieved 100\% uptime for three years through boring, reliable engineering. Side gig that proved comprehensive testing beats clever code.

\textbf{What Went Right (and Wrong):}
\begin{itemize}
\item \textbf{Quality Pays Off:} As sole developer, invested heavily in testing (70\% coverage) and automation. Quality investment accelerates velocity---tests caught issues before production did.
\item \textbf{Architectural Honesty:} The middleware was unnecessary. AWS Lambda with eventing would've been cheaper, simpler, faster. Sometimes the best architecture is the one you don't build.
\item \textbf{External Dependencies:} Well-designed middleware couldn't fix Saloodo's database inconsistencies (malformed JSON, wrong types). Your quality standards don't matter when dependencies are chaos.
\item \textbf{Performance Optimization:} Reduced SELECT times 80\% through SQL EXPLAIN analysis. Databases are fast when you use them correctly. Shocking but true.
\end{itemize}

\vspace{4pt}

\subsection*{Cloud Consultant at Trusted Shops (Aug 2018 -- Oct 2019)}
\textit{Microservices Disillusionment $\cdot$ Where I Learned to Question Architectural Dogma}

Built subscription management and message delivery systems on AWS Lambda. First encounter with real microservices architecture revealed the gap between conference talks and operational reality.

\textbf{The Microservices Hangover:}
\begin{itemize}
\item \textbf{Microservices Reality Check:} Simple changes required modifications across many services. Lack of testing and automation made maintenance hell. Learned that microservices without discipline is distributed monolith with network latency.
\item \textbf{Architecture Skepticism Born:} Since this project, prefer Self-contained Systems over microservices. Team autonomy matters more than service boundaries. (Later validated at Union Investment.)
\item \textbf{Integration Theater:} Seamlessly integrated Salesforce, Zuora, and Twilio. ``Seamlessly'' means ``after reading terrible API docs and handling edge cases they forgot to document.''
\item \textbf{Contract Testing Evangelism:} Led Pact training across teams. When services can't talk to each other, contract testing prevents integration surprises. Prevention beats debugging distributed systems.
\end{itemize}

\newpage

\section*{Earlier Professional Experience: The Foundation Years}

\subsection*{Senior Engineer at toom Baumarkt (Oct 2016 -- Sep 2017)}
\textit{via tarent solutions $\cdot$ Where I Met Kubernetes and Became a Skeptic}

Developed microservices for home improvement retail platform following ``you build it, you run it'' principle. Led cross-team quality initiatives while learning that Kubernetes is a tool for platform providers, not application developers.

\textbf{Lessons in Complexity:}
\begin{itemize}
\item \textbf{Kubernetes Skepticism:} First exposure to Kubernetes. Still seems overly complex for both ops and dev teams running individual applications. Great for platform providers, overkill for most applications---especially in hyperscaler environments with better alternatives.
\item \textbf{Quality Leadership:} Led weekly cross-team quality initiatives across 4 dev teams. Learned that consistent standards matter more than individual brilliance.
\item \textbf{Performance Optimization:} Accelerated e2e tests by 70\%, optimized Java memory through GC tuning. Because fast feedback loops enable fast learning.
\item \textbf{Testing Excellence:} Introduced integration tests following Maven lifecycle and test pyramid principles. Testing isn't overhead---it's how you move fast without breaking everything.
\end{itemize}

\vspace{4pt}

\subsection*{PHP Developer \& Head of Development at wer-kennt-wen (Aug 2007 -- Jun 2014)}
\textit{Social Network (9M+ users) $\cdot$ Where Leadership Lessons Were Expensive}

Progressed from PHP developer to Head of Development at one of Germany's largest social networks. Led architecture modernization and learned that technical competence doesn't automatically translate to leadership ability.

\textbf{Career-Defining Failures:}
\begin{itemize}
\item \textbf{Leadership Too Soon:} Accepted Head of Development role without management experience. Both I and company assumed technical skills = leadership ability. Wrong. Initially underestimated human factors, focusing only on technical aspects. Through training and reflection, evolved toward laissez-faire style that suited both personality and team.
\item \textbf{Sphinx over ElasticSearch:} Chose familiar Sphinx over ``exotic'' ElasticSearch. Conservative choice proved suboptimal---platform later migrated to ElasticSearch anyway. Lesson: familiarity bias kills innovation. Better choice would've been better long-term.
\item \textbf{Architecture Modernization Success:} Led transformation from procedural to OOP with MVC patterns, migrated group and messaging systems to Zend Framework. Sometimes getting architecture right matters more than getting it fast.
\item \textbf{Scale Learning:} Successfully scaled for 9M+ users through partitioning, Memcache, and database optimization. Scale reveals every architectural mistake exponentially.
\end{itemize}

\vspace{4pt}

\subsection*{Early Freelancing (Jan 2014 -- Aug 2016)}
\textit{Multiple Companies $\cdot$ CI/CD Evangelism and AWS Migration}

Short-term contracts at YOOCHOOSE, Xsite, freshcells, fotocommunity, billiger-mietwagen.de, AffiliCon, dailypresent. Implemented Jenkins CI/CD pipelines, established code quality standards, and led AWS migrations.

\textbf{Learning to Consult:}
Implemented CI/CD for 6+ companies, learning that most organizations need process discipline more than technical brilliance. Automated deployment pipelines, established quality frameworks, and discovered that Docker integration solves more problems than it creates (unlike Kubernetes).

\vspace{6pt}

\section*{Personal Projects: Where I Experiment So You Don't Have To}

\subsection*{ConversationalAI4J (2025)}
Voice-enabled conversational AI using local models. Because privacy matters and cloud APIs are expensive when you're just talking to yourself.

\textbf{Technologies:} Java 21, Spring Boot, Ollama/LLaMA, Whisper.cpp, Piper TTS, WebSocket\\
\textbf{What I Learned:} Native bindings are fast but debugging them is slow. Local AI is powerful but hungry for RAM.

\subsection*{SchulteDev Portfolio Repository (2025)}
AI-powered CV generation using Claude 4, GitHub Actions automation, and LaTeX. This very document was generated by a script that reads career data and produces multiple CV formats.

\textbf{Technologies:} Node.js, Claude 4 API, LaTeX, GitHub Actions\\
\textbf{What I Learned:} AI is great at formatting my failures consistently. Automation means I can generate embarrassing documents faster.

\subsection*{AI Document Processing for Tax Software (Jan--May 2025)}
CLI application automating document capture using Azure Document Intelligence and SteuerSparErklärung tax software integration.

\textbf{Technologies:} Golang, Azure Document Intelligence, SQLite\\
\textbf{What I Learned:} AI document processing works great until it encounters handwriting or creative accounting.

\subsection*{Atlassian Bamboo Server Add-ons (Jun 2016 -- Dec 2023)}
Commercial Marketplace add-ons integrating code quality metrics into Bamboo CI/CD. Sustainable 7-year business with 83 customers across 15+ countries.

\textbf{Technologies:} Java, Atlassian SDK, OSGi, jQuery, DataTables\\
\textbf{What I Learned:} First-mover advantage matters. Sustainable business requires boring reliability over exciting features.

\vspace{2pt}

\textit{Also created: Jenkins↔GitHub integration PoC (2018), Crypto-Arbitrage Trading Bot (2015)---proof that not every idea needs to become a business.}

\vfill

\newpage

\section*{Education: The Extended Program}

\subsection*{Diplom in Computer Science -- University of Koblenz-Landau (2004--2012)}
\textit{Graduated with Distinction (magna cum laude equivalent) $\cdot$ Focus: Computational Visualistics \& Software Engineering}

The German Diplom (equivalent to Master of Science) took longer than standard duration because I worked part-time and sometimes full-time as software developer starting in 2007. Delayed studies but provided valuable practical experience alongside academic learning. Turns out you can't debug production issues and study for exams simultaneously---who knew?

\textbf{Extended Duration Lessons:}
\begin{itemize}
\item Real-world experience beats coursework timelines
\item Working full-time while studying is possible but inadvisable
\item Academic deadlines are flexible, production deadlines are not
\item Graduated with distinction anyway because practical experience + theory = actual understanding
\end{itemize}

\vspace{4pt}

\subsection*{Abitur -- Gymnasium Haren (Ems) (2003)}
\textit{Advanced Courses: Mathematics and Physics}

Had first contact with IT in sixth grade typing course---last PC contact at school. Self-taught PHP (2000--2003) using reference books, created homepages and forums for friend groups. Discovered computer games, decided to study Computer Visualistics. Typical origin story.

\vspace{6pt}

\section*{Certifications: The Resume Padding}

\begin{itemize}
\item \textbf{Java SE 8 TechCheck (IKM):} 93/100 score, 88th percentile (June 2018)---proof I can pass tests about Java
\item \textbf{StackOverflow:} 4000+ reputation points---proof I can answer questions about Java
\item \textbf{GitHub:} Open-source contributor---proof I write code others can judge
\end{itemize}

\vspace{6pt}

\section*{Additional Skills: The Miscellaneous Category}

\textbf{Languages:} German (native), English (fluent)---can explain architectural mistakes in two languages

\textbf{Agile Methodologies:} Scrum, Kanban, SAFe---survived all three without becoming dogmatic

\textbf{Leadership:} Team building, technical mentoring, stakeholder management---learned that managing people is harder than managing services

\textbf{Communication:} Regular presentations to business and IT stakeholders---learned to translate ``distributed event-sourced architecture'' into ``messages between services''

\vspace{6pt}

\section*{Career Breaks: The Intentional Gaps}

\subsection*{Parental Leave (Apr 2020 -- Sep 2021)}
During COVID-19 pandemic lockdowns. Combined family time with industry-wide project slowdown affecting freelance opportunities. Learned that debugging toddler tantrums is harder than debugging distributed systems.

\subsection*{Personal Travel Break (Aug 2015 -- Dec 2015)}
Extended travel period exploring different countries and cultures. Learned that some problems require distance and perspective, not just better architecture.

\subsection*{Military Service (Aug 2003 -- Mar 2004)}
Mandatory military service, German Air Force. Completed basic training at Nassau-Dietz-Barracks (Budel, NL), transferred to Hopsten airbase (Rheine, DE). Learned that hierarchy and command structures are terrible for software development.

\vfill

\begin{center}
\small\textit{This Anti-CV was generated using AI (Claude 4) from comprehensive career data,}\\
\textit{proving that even artificial intelligence can format human failure consistently.}\\[4pt]
\textit{All failures, lessons learned, and self-deprecating humor are factually accurate.}\\
\textit{The embarrassment is genuine and unenhanced.}
\end{center}

\end{document}