\documentclass[10pt,a4paper]{article}
\usepackage[left=1.5cm,right=1.5cm,top=1.5cm,bottom=1.5cm]{geometry}
\usepackage{enumitem}
\usepackage{titlesec}
\usepackage{xcolor}
\usepackage{fontspec}
\usepackage{hyperref}
\usepackage{multicol}
\usepackage{tabularx}

\setmainfont{Latin Modern Roman}
\setsansfont{Latin Modern Sans}
\setmonofont{Latin Modern Mono}

\definecolor{darkblue}{RGB}{0,50,100}
\definecolor{gray}{RGB}{100,100,100}

\titleformat{\section}{\Large\bfseries\sffamily\color{darkblue}}{}{0em}{}[\titlerule]
\titleformat{\subsection}{\large\bfseries\sffamily\color{darkblue}}{}{0em}{}
\titleformat{\subsubsection}{\normalsize\bfseries\sffamily\color{darkblue}}{}{0em}{}

\setlength{\parindent}{0pt}
\setlength{\parskip}{3pt}

\hypersetup{
    colorlinks=true,
    linkcolor=darkblue,
    urlcolor=darkblue,
    citecolor=darkblue
}

\begin{document}

\begin{center}
{\Huge\bfseries\sffamily\color{darkblue} MARKUS SCHULTE}\\[5pt]
{\Large\sffamily Anti-CV: A Professional Guide to Spectacular Learning Experiences}\\[8pt]
{\large\sffamily Freelance Cloud Architect \& Professional Mistake Maker}\\[5pt]
\textcolor{gray}{Cologne, Germany | \href{mailto:mail@schulte-development.de}{mail@schulte-development.de} | +49 178 7217768}\\
\textcolor{gray}{\href{https://schulte-development.de}{schulte-development.de} | \href{https://github.com/SchulteDev}{GitHub} | \href{https://linkedin.com/in/markus-schulte}{LinkedIn}}
\end{center}

\section{Professional Summary of Magnificent Failures}

\textbf{Cloud Architect \& Software Engineer} with 18+ years experience in creative problem-solving, over-engineering solutions, and discovering what doesn't work. Specialized in building complex systems that teach valuable lessons about simplicity.

\textbf{Core Competencies in Failure:}
• \textbf{Over-Architecture:} Expert in creating microfrontend frameworks when a simple webpage would suffice
• \textbf{Technology Skepticism:} Professional pessimist regarding Kubernetes, microservices, and other "silver bullets"
• \textbf{Leadership Learning:} Mastered the art of accepting management roles without management experience
• \textbf{Tool Proliferation:} Skilled in introducing 47 different tools to solve problems that could be solved with 3

\textbf{Greatest Achievements in Learning:}
• Successfully convinced Union Investment to let me build a microfrontend framework for 80,000 users
• Took 7+ years to complete a university degree that should take 5 years
• Discovered that microservices create more problems than they solve (the hard way)
• Built middleware that nobody asked for but everyone needed

\begin{center}
\begin{tabularx}{\textwidth}{|X|X|X|X|}
\hline
\textbf{Failure Category} & \textbf{Years of Experience} & \textbf{Lessons Learned} & \textbf{Recovery Rate} \\
\hline
Over-Engineering & 18+ & Simplicity is harder than complexity & 95\% \\
\hline
Technology Hype & 11+ & Not every trend is worth following & 90\% \\
\hline
Architecture Complexity & 8+ & KISS principle exists for a reason & 85\% \\
\hline
Team Leadership & 6+ & People skills matter more than code & 80\% \\
\hline
\end{tabularx}
\end{center}

\section{Recent Professional Learning Experiences}

\subsection{November 2024 — Present: Cloud Architect at LBBW}
\textit{Currently attempting to modernize a major German bank's IT infrastructure}

\textbf{Current Challenges:}
• Working with a platform called "Skywalker" that's not yet in production (what could go wrong?)
• Responsible for 10 of 30 applications in a transformation project with 80 people (no pressure)
• Trying to convince banking IT that cloud migration is worth the risk
• Learning that enterprise banking moves at the speed of continental drift

\textbf{Key Learning Experiences:}
• Discovering that "production-ready" doesn't mean "production-used"
• Finding out that R-methodology assessments can reveal uncomfortable truths about legacy systems
• Realizing that Azure Calculator estimates and reality have a complex relationship
• Understanding that "Phoenix" transformations sometimes involve more ashes than rising

\subsection{October 2021 — December 2024: Cloud Architect at Union Investment}
\textit{Three years of magnificent over-engineering for 80,000 users}

\textbf{The Grand Vision:} Transform a simple B2B platform into a distributed masterpiece using Self-contained Systems, microfrontends, and event-driven architecture.

\textbf{Notable Achievements in Complexity:}
• Built a microfrontend framework called "microfrontend-toolkit" because clearly the world needed another one
• Convinced management that 8 domain boundaries were necessary for what was essentially a product catalog
• Created an event-driven search system using Azure Event Hubs and Algolia when a database query might have sufficed
• Achieved 1000+ tests with 10-minute build times (because waiting builds character)
• Introduced so many tools that the onboarding process became a multi-week journey

\textbf{Lessons Learned:}
• Sometimes a monolith is just a monolith waiting to be over-engineered
• Event-driven architecture is event-driven complexity
• Web Components are great, but explaining them to stakeholders is an art form
• Azure Static Web Apps are fantastic until you need to do anything complex

\textbf{The Reality Check:} Despite all the complexity, the platform actually works well and serves as a corporate template. Sometimes over-engineering accidentally creates good architecture.

\subsection{December 2021 — November 2024: Middleware Engineer at Saloodo! (DHL)}
\textit{Solo mission: Sync 10 million records while learning Golang}

\textbf{The Challenge:} Build middleware nobody asked for but everyone needed, using a language I'd never used professionally.

\textbf{Creative Solutions:}
• Chose this part-time project specifically to learn Golang in production (learning by fire)
• Built a system that runs in endless loops because apparently that's how you do middleware
• Achieved 100\% uptime through sheer stubbornness and comprehensive testing
• Discovered that Saloodo's database contains creative interpretations of data types
• Implemented retry logic that would make a cardiac patient jealous

\textbf{Lessons Learned:}
• Sometimes the best way to learn a language is to bet someone else's production system on it
• Postgres inconsistencies are a feature, not a bug
• 2,776 lines of code can handle 10 million records if you're creative enough
• Dockertest is your friend when you're the only developer

\subsection{August 2018 — October 2019: Cloud Consultant at Trusted Shops}
\textit{Where I discovered microservices aren't magic}

\textbf{The Awakening:} First encounter with real microservices architecture in the wild.

\textbf{Shocking Discoveries:}
• Simple changes required modifications across 17 different services
• Testing was more complex than the actual business logic
• Deploy ordering became a strategic puzzle game
• Integration challenges grew exponentially with service count

\textbf{Technologies Mastered Through Suffering:}
• AWS Lambda (when you want to make simple things complicated)
• Terraform (Infrastructure as Code as Confusion)
• Pact contract testing (teaching services to talk to each other)
• Salesforce/Zuora integration (because nothing says fun like third-party APIs)

\textbf{The Revelation:} Became professionally skeptical of microservices as the solution to all problems. Now prefer Self-contained Systems for maintaining sanity.

\section{Earlier Learning Adventures}

\subsection{October 2016 — September 2017: Senior Engineer at toom Baumarkt}
\textit{Where I met Kubernetes and immediately became suspicious}

\textbf{First Encounters:}
• Docker: "This seems useful!"
• Kubernetes: "This seems unnecessarily complex for running a simple application"
• Microservices: "Why do we need 5 services to display a product page?"

\textbf{Quality Leadership Adventures:}
• Led quality initiatives across 4 teams (herding cats, but technical cats)
• Introduced so many testing tools that pull requests became archaeological expeditions
• Achieved 70\% performance improvement in tests (because faster failures are better failures)

\textbf{The Kubernetes Skepticism Origin Story:} Spent weeks learning Kubernetes only to realize it's a tool for operations teams, not application developers. Became lifelong advocate for "use the right tool for the job."

\subsection{August 2007 — June 2014: PHP Developer at werkenntwen}
\textit{9 million users and two major career lessons}

\textbf{Lesson 1: Leadership Without Experience (2008-2010)}
• Accepted Head of Development role based purely on technical skills
• Discovered that managing code is easier than managing people
• Learned that "technical competence = leadership ability" is a dangerous equation
• Evolved from technical dictator to collaborative leader (character development arc)

\textbf{Lesson 2: Technology Choices Have Consequences}
• Chose Sphinx over ElasticSearch because it seemed "more reliable"
• Watched as ElasticSearch became the industry standard
• Later migrated to ElasticSearch anyway (validation of the road not taken)
• Learned that conservative technology choices aren't always safe choices

\textbf{Scale Learning Experience:}
• Supported 9+ million users with PHP and MySQL (because we were young and fearless)
• Implemented creative caching strategies that would make Redis jealous
• Discovered that social networks are just elaborate systems for moving data around

\subsection{Personal Projects: The Laboratory of Learning}

\textbf{2025: SchulteDev Portfolio Repository}
• Built an AI-powered CV generator because apparently I needed Claude to write my resume
• Created a monorepo for professional branding (because one repo wasn't enough)
• Implemented GitHub Actions that are more complex than the actual content they build

\textbf{2025: AI Document Processing for Tax Software}
• Built a CLI tool using Azure Document Intelligence for tax documents
• Discovered that AI document processing is really good at being confidently wrong
• Learned that Golang is actually quite pleasant for CLI applications

\textbf{2018: Jenkins ↔ GitHub Integration PoC}
• Attempted to solve a problem that GitHub Actions would solve better years later
• Built multi-tenant Jenkins setup that was impressive but ultimately unnecessary
• Learned that being early to a solution is sometimes indistinguishable from being wrong

\textbf{2015: Crypto-Arbitrage Trading Bot}
• Built Bitcoin trading algorithms during the early crypto days
• Discovered that arbitrage opportunities disappear faster than you can code them
• Learned that machine learning and trading are both harder than they look

\section{Education and Skill Development}

\subsection{University of Koblenz-Landau (2004-2012)}
\textbf{Diploma in Computer Science} - Graduated with distinction after 8 years

\textbf{The Extended University Experience:}
• Took 8 years to complete a 5-year degree (because real-world experience was calling)
• Worked full-time as a developer while studying (time management through suffering)
• Learned that practical experience and academic theory have a complicated relationship
• Discovered that "Computational Visualistics" is a fancy way of saying "Computer Graphics"

\textbf{Key Learning:} Sometimes taking longer to finish something means you understand it better.

\subsection{Professional Certifications in Failure}
• \textbf{StackOverflow:} 4000+ reputation points (proof that I've encountered enough problems to help others)
• \textbf{Java SE 8 TechCheck:} 93/100 score (proof that I can pass tests, even if I make mistakes in practice)
• \textbf{GitHub:} Multiple repositories of educational code (some of which even work)

\subsection{Core Skills in Creative Problem-Solving}
\textbf{Programming Languages:} Java (18y of mistakes), Golang (4y of learning), PHP (8y of evolution), TypeScript (ongoing relationship)

\textbf{Architecture Patterns:} Microservices (skeptical), Self-contained Systems (preferred), Event-driven (with caution), Monoliths (underrated)

\textbf{Cloud Platforms:} AWS (7y of bills), Azure (4y of documentation), GCP (occasional visitor)

\textbf{Leadership Skills:} Team building (learned through trial and error), Product ownership (accidental expertise), Stakeholder management (ongoing education)

\textbf{Quality Assurance:} Contract testing (Pact evangelist), Test automation (comprehensive to a fault), Code quality (SonarQube enthusiast)

\subsection{Career Breaks and Life Lessons}
\textbf{April 2020 — September 2021:} Parental Leave (discovered that managing toddlers is harder than managing developers)

\textbf{August 2015 — December 2015:} Personal Travel (learned that the world is bigger than our code)

\textbf{August 2003 — March 2004:} Military Service (learned that following orders is different from leading people)

\section{Professional Philosophy and Lessons Learned}

After 18+ years of professional mistakes, I've developed a philosophy: \textbf{"Every failure is a feature of the learning process."} My career has been built on the principle that the best way to learn is to try things that might not work, in environments where the stakes are real.

\textbf{Core Beliefs:}
• Simple solutions are harder to design than complex ones
• The right tool for the job is usually simpler than you think
• People problems are harder than technical problems
• Documentation is written for your future self, who will have forgotten everything
• The best architectures are the ones that can be explained to a new team member in 30 minutes

\textbf{What I've Learned About Technology:}
• Kubernetes is powerful but often overkill
• Microservices solve some problems while creating others
• Event-driven architecture is event-driven complexity
• The cutting edge is called that because it cuts
• Legacy systems survive because they work, not because they're beautiful

\textbf{What I've Learned About Leadership:}
• Technical skills and leadership skills are different skill sets
• The best technical solution isn't always the best business solution
• Teaching others is the best way to learn yourself
• Admitting mistakes builds more trust than hiding them
• Sometimes the most important thing a leader can do is get out of the way

\textbf{Current Mission:} Helping organizations navigate the space between "what's possible" and "what's practical," using hard-won experience to avoid the exciting mistakes I've already made.

\vfill

\begin{center}
\textcolor{gray}{\small This anti-CV was generated using AI assistance (Claude 4) to maintain appropriate self-deprecating humor while preserving professional accuracy. The failures described are real, the lessons learned are valuable, and the sense of humor is professionally therapeutic.}
\end{center}

\end{document}