\documentclass[10pt,a4paper]{article}
\usepackage[margin=1.5cm]{geometry}
\usepackage{fontspec}
\usepackage{hyperref}
\usepackage{enumitem}
\usepackage{tabularx}
\usepackage{xcolor}

\setmainfont{Latin Modern Roman}
\setsansfont{Latin Modern Sans}

\definecolor{darkgray}{gray}{0.3}
\hypersetup{colorlinks=true, urlcolor=blue, linkcolor=blue}

\setlength{\parindent}{0pt}
\setlength{\parskip}{4pt}
\pagestyle{empty}

\newcommand{\sectiontitle}[1]{%
  \vspace{8pt}%
  {\large\bfseries\textsf{#1}}\\[4pt]%
  \textcolor{darkgray}{\rule{\textwidth}{0.5pt}}%
  \vspace{4pt}%
}

\begin{document}

% PAGE 1
\begin{center}
{\LARGE\bfseries\textsf{Markus Schulte}}\\[4pt]
{\large\textsf{Anti-CV: Things I Learned the Hard Way}}\\[8pt]
\textsf{Cloud Architect \& Software Engineer | Cologne, Germany}\\[2pt]
\href{mailto:mail@schulte-development.de}{mail@schulte-development.de} | +49 178 7217768 | \href{https://schulte-development.de}{schulte-development.de}\\
\href{https://linkedin.com/in/markus-schulte}{LinkedIn} | \href{https://github.com/SchulteDev}{GitHub} | \href{https://stackoverflow.com/users/1645517/markus-schulte}{StackOverflow}
\end{center}

\vspace{6pt}

\sectiontitle{Professional Summary (Or: How I Stopped Worrying and Learned to Love Legacy Code)}

19+ years building software, 12+ years as freelancer. Specializing in cloud-native transformations, which is consultant-speak for ``helping companies migrate away from technologies I once enthusiastically recommended.'' Expert in Java, Golang, Azure, AWS---and in recognizing which architectural decisions will haunt you at 2 AM.

\textbf{Core Competency:} Making expensive mistakes so you don't have to. Pattern recognition developed through repeatedly choosing technologies that seemed brilliant until production.

\vspace{6pt}

\sectiontitle{Key Anti-Achievements}

\begin{itemize}[leftmargin=*, itemsep=2pt, topsep=0pt]
\item \textbf{Kubernetes Skeptic:} First encounter with K8s at toom Baumarkt (2016--2017). Still too complex for most use cases. Became fundamentally skeptical of tools that require a dedicated team to operate what should be simple applications. If your app needs Kubernetes, maybe your app is the problem.
\item \textbf{Microservices Disillusionment:} First real microservices architecture at Trusted Shops (2018--2019). Discovered that ``simple changes require modifications across many services'' is a feature, not a bug. Converted to Self-contained Systems religion thereafter.
\item \textbf{Leadership Reality Check:} Accepted Head of Development role at wer-kennt-wen (2008) without management experience. Assumed technical competence equals leadership ability. Narrator: \textit{It didn't}. Learned that ``laissez-faire'' is French for ``I'm still figuring this out.''
\item \textbf{Architectural Overengineering:} Built middleware at Saloodo (2021--2024) when AWS Lambda would have been simpler, cheaper, and faster. Sometimes the boring solution is the right solution.
\item \textbf{Conservative Technology Choices:} Selected Sphinx over ElasticSearch at wer-kennt-wen (2009) due to familiarity. Platform later migrated to ElasticSearch anyway. Lesson: ``Exotic'' features eventually become industry standard.
\item \textbf{Team Composition Learning:} Had to remove team members at Union Investment despite integration attempts. Discovered that team fit matters more than individual talent. Still uncomfortable, still necessary.
\end{itemize}

\vspace{6pt}

\sectiontitle{Technical Expertise (Mistakes Made In)}

\begin{tabularx}{\textwidth}{@{}lX@{}}
\textbf{Languages} & Java (19y, mother tongue), Golang (5y, preferred), PHP (8y, foundation), TypeScript/JavaScript \\
\textbf{Cloud} & Azure (Functions, AKS, Event Hubs), AWS (EKS, Lambda, SQS), GCP (GKE, Functions) \\
\textbf{Architecture} & Microservices (skeptical), Self-contained Systems (converted), Event-driven, Monoliths (nostalgic) \\
\textbf{Databases} & PostgreSQL, MySQL, Oracle, MS-SQL, Redis --- all excellent places to store technical debt \\
\textbf{DevOps} & Docker, Kubernetes (reluctantly), Terraform, Jenkins, GitHub Actions, CI/CD pipelines \\
\textbf{Testing} & JUnit, Mockito, Pact, Selenium, Playwright --- extensive experience writing tests nobody reads \\
\textbf{Frameworks} & Spring Boot, Quarkus, Micronaut, Node.js, Lit, Web Components \\
\textbf{Leadership} & Team building (5--9 developers), Technical mentoring, Agile coaching, Difficult conversations \\
\end{tabularx}

\vspace{6pt}

\sectiontitle{Career Philosophy}

\textit{``Crafting solutions that thrive today and tomorrow''} --- though ``tomorrow'' often arrives sooner than expected, usually with a production outage. Specialize in identifying which best practices actually matter and which are just FAANG cargo-culting.

\vfill

\newpage

% PAGE 2
\sectiontitle{Recent Professional Experience (Or: My Greatest Hits of Humility)}

\textbf{Cloud Architect | LBBW (Landesbank Baden-W\"urttemberg)} \hfill \textit{Nov 2024--Oct 2025}\\
\textit{Freelance | Remote | ``Phoenix'' IT Modernization}

Major German state bank migrating to Azure. Tasked with assessing hundreds of legacy applications for cloud readiness. The Skywalker platform was production-ready but production-unused during my tenure---excellent opportunity to identify gaps in something nobody was using yet.

\begin{itemize}[leftmargin=*, itemsep=2pt, topsep=0pt]
\item \textbf{Legacy Application Archaeology:} Cataloged hundreds of applications with R-methodology. Discovered that ``hundreds of legacy applications'' means ``hundreds of unique ways to do the same thing.'' Each requiring individualized migration assessment because someone in 2003 thought Oracle Forms was the future.
\item \textbf{Platform Gap Analysis:} Identified gaps in production-unused Skywalker platform. Diplomatically communicated that ``production-unused'' and ``production-ready'' might not mean the same thing.
\item \textbf{Azure Architecture Design:} Created migration strategies for Spring Boot, Tomcat, and IIS applications. Learned that ``lift and shift'' sounds simple until you're shifting applications nobody remembers writing.
\item \textbf{Stakeholder Translation:} Translated technical architectures to business committees. Mastered the art of explaining why ``just move it to the cloud'' isn't actually simple without making anyone feel bad about asking.
\end{itemize}

\textbf{Technologies:} Azure (AKS, Functions), Spring Boot, Tomcat, IIS, Oracle, MS-SQL, PostgreSQL, PowerPoint (surprisingly important)

\vspace{8pt}

\textbf{Cloud Architect | Union Investment} \hfill \textit{Oct 2021--Dec 2024}\\
\textit{Freelance | Remote | 3+ Year Transformation Journey}

Leading German asset management company. Modernized InvestmentWelt B2B platform from Liferay monolith to Self-contained Systems architecture. Three years of comprehensive transformation teaching me everything about what works, what doesn't, and what seemed like a good idea at the time.

\begin{itemize}[leftmargin=*, itemsep=2pt, topsep=0pt]
\item \textbf{Microfrontend Framework:} Built custom Web Components toolkit enabling autonomous team development. Chose Web Components after initial PoC, saving 60 person-days by \textit{not} exploring every possible alternative. Sometimes ``bias for action'' beats ``analysis paralysis.''
\item \textbf{Cross-Domain Search:} Designed centralized search consuming event streams from 8 distributed domains. Initially seemed impossible; solution was embarrassingly simple Azure Event Hubs Pub/Sub. Sometimes the straightforward approach beats the clever one.
\item \textbf{Algolia vs ElasticSearch:} Deep evaluation led to Algolia. Learned from wer-kennt-wen Sphinx mistake---chose the more advanced option this time. Still second-guessing myself, but in a healthier way.
\item \textbf{Team Building \& Difficult Decisions:} Built cross-functional team from scratch. Had to remove members who didn't fit despite integration attempts. Learned that ``team fit'' isn't HR buzzwords; it's the difference between shipping and suffering.
\item \textbf{360° Feedback Highlights:} ``Excellent technical expertise'' (reassuring). ``Consider pace of technical explanations for less technical stakeholders'' (working on it). ``Strong efforts to advance architectural topics'' (sometimes too strong, apparently).
\item \textbf{Production Success:} 99.9\% uptime, deployment time reduced from days to 30 minutes. Established SCS pattern as corporate standard. Proved that applying IT best practices brings business value, despite requiring actual effort.
\end{itemize}

\textbf{Technologies:} Azure (Static Web Apps, Functions, Event Hubs, AD B2C), Web Components, Lit, TypeScript, Java, Algolia, Terraform, Azure DevOps

\vspace{8pt}

\textbf{Middleware Engineer | Saloodo! (DHL Group)} \hfill \textit{Dec 2021--Nov 2024}\\
\textit{Freelance | Part-Time | Remote}

Sole developer for Saloodo-to-Salesforce data synchronization. Built robust Golang middleware handling 10M records. Architecturally unnecessary---AWS Lambda would have been simpler---but excellent Golang learning opportunity.

\begin{itemize}[leftmargin=*, itemsep=2pt, topsep=0pt]
\item \textbf{Architectural Simplicity Lesson:} Middleware was unnecessary complexity. Saloodo ran on AWS; integrated Lambda solution would've been cheaper, simpler, faster. Built solid middleware anyway. 100\% uptime over 3 years, so technically successful despite strategic inefficiency.
\item \textbf{External Dependencies:} Saloodo's database contained malformed JSON and incorrect types. Learned that well-designed middleware suffers from upstream quality issues. Cannot fix systemic problems with local excellence.
\item \textbf{Solo Ownership:} Sole developer for critical component. Implemented comprehensive testing (70\% coverage, 5-second execution) and monitoring because production outages are lonely at 2 AM.
\item \textbf{Quality Standards Impact:} High quality standards with comprehensive testing delivered speed improvements that quickly paid for themselves. Quality investment accelerates development velocity---but only if you're disciplined about it.
\end{itemize}

\textbf{Technologies:} Golang, PostgreSQL, Salesforce Bulk API, Docker, Kubernetes, ELK Stack, Dynatrace

\vspace{8pt}

\textbf{Cloud Consultant | Trusted Shops} \hfill \textit{Aug 2018--Oct 2019}\\
\textit{Freelance | Cologne}

E-commerce trust provider requiring legacy JavaEE6 modernization. First real microservices architecture experience. Discovered lived disadvantages: simple changes require modifications across many services, lack of testing makes maintenance difficult.

\begin{itemize}[leftmargin=*, itemsep=2pt, topsep=0pt]
\item \textbf{Microservices Reality:} First encounter with real microservices architecture. Surprised by the disadvantages. IT architecture not open to change. Since this project, skeptical of microservices as overarching architecture. Prefer Self-contained Systems (as later vindicated at Union Investment).
\item \textbf{AWS Lambda \& Micronaut:} Built serverless solutions with comprehensive Terraform. Created reusable Lambda templates. Learned that serverless doesn't mean effortless, just differently effortful.
\item \textbf{Contract Testing:} Led Pact framework training across multiple teams. Discovered that preventing integration issues requires convincing developers to write more tests. Mixed success.
\item \textbf{Integration Complexity:} Seamless Salesforce, Zuora, Twilio integration. Learned that ``seamless'' means ``works eventually after extensive debugging.''
\end{itemize}

\textbf{Technologies:} Java, Spring Boot, Micronaut, AWS (Lambda, EKS, CloudFormation), Terraform, Pact, Salesforce, Zuora

\vfill

\newpage

% PAGE 3
\sectiontitle{Earlier Professional Experience (Where the Learning Really Happened)}

\textbf{Senior Engineer \& DevOp | toom Baumarkt GmbH} \hfill \textit{Oct 2016--Sep 2017}\\
\textit{via tarent solutions GmbH | Freelance | Bonn}

8-member Scrum team following ``you build it, you run it'' principle. Additional quality leadership role across 4 teams. First encounter with docker-compose and Kubernetes.

\begin{itemize}[leftmargin=*, itemsep=2pt, topsep=0pt]
\item \textbf{Kubernetes Skepticism Origin Story:} First encounter with K8s. Seemed overly complex for both operations and development teams. Became fundamentally skeptical---remains too complex for individual applications. Tool for IT operations providers, not simple applications. Hyperscaler-managed services are better for most use cases.
\item \textbf{Testing Excellence:} Achieved 70\% performance improvement in e2e tests through optimization. Introduced integration tests following Maven lifecycle. Discovered that fast tests get run; slow tests get skipped.
\item \textbf{Quality Leadership:} Led weekly cross-team quality initiatives across 4 development teams. Learned that establishing standards is easier than enforcing them.
\item \textbf{Performance Optimization:} Garbage Collection tuning reduced Java microservice memory consumption. Sometimes the JVM knows better, sometimes you need to convince it otherwise.
\end{itemize}

\textbf{Received formal technical and organizational recommendations from tarent leadership.}

\vspace{8pt}

\textbf{PHP Developer \& Head of Development | wer-kennt-wen (werkenntwen GmbH)} \hfill \textit{Aug 2007--Jun 2014}\\
\textit{Permanent Position | One of Germany's largest social networks (9M+ members)}

Career path: PHP Developer (2007--2008), Head of Development (2008--2010), Backend Developer (2010--2014). Company acquired by RTL, ceased operations June 2014.

\begin{itemize}[leftmargin=*, itemsep=2pt, topsep=0pt]
\item \textbf{Leadership Without Experience:} Accepted Head of Development role without management experience. Both company leadership and I assumed technical competence would translate. Narrator: \textit{It required actual leadership skills}. Initially underestimated human factor, focusing primarily on technical aspects. Eventually evolved toward laissez-faire style after training and reflection.
\item \textbf{Sphinx vs ElasticSearch:} Selected Sphinx over ElasticSearch due to familiarity (2009). Conservative choice proved suboptimal---platform later migrated to ElasticSearch during my return as Backend Developer. Validated that innovative choice would've been better long-term. Learned to distinguish ``exotic'' from ``advanced.''
\item \textbf{Architecture Modernization:} Led transformation from procedural to OOP with MVC patterns. Migrated group system and messaging to Zend Framework. Successfully scaled to 9M+ users. Technical execution succeeded despite leadership growing pains.
\item \textbf{Security \& Performance:} Implemented comprehensive security (XSS, CSRF, SQL-injection prevention). Deployed table partitioning and Memcache. Reduced page load times 60\%. Technical foundation was solid; organizational lessons came harder.
\end{itemize}

\vspace{8pt}

\textbf{Additional Experience (The Formative Years):}

\textit{Senior JavaEE Developer | Allianz GDF} (via SinnerSchrader, Nov 2017--Mar 2018): CloudFoundry on AWS, REST/HATEOAS APIs. Learned that comprehensive API documentation becomes company standard when nobody else wants to write docs.

\textit{Cloud Consultant | Sedo GmbH} (Jan--Mar 2020): Consolidated Spring Boot/Camel services into Quarkus. Led Pact contract testing workshops. Discovered that consolidating services is satisfying; teaching contract testing is challenging.

\textit{Early Freelancing} (Jan 2014--Aug 2016): YOOCHOOSE (Spring/AWS), Xsite (REST APIs), freshcells (Jenkins/Docker), fotocommunity (Bamboo CI), billiger-mietwagen.de (infrastructure modernization), AffiliCon (Jenkins setup), dailypresent (PHP/AWS). Implemented CI/CD pipelines for 6+ companies. Learned that everyone wants DevOps; few want to do the work.

\vspace{8pt}

\sectiontitle{Personal Projects (Because Professional Projects Aren't Enough Mistakes)}

\textbf{ConversationalAI4J} (2025): Voice-enabled conversational AI using local models. Java library for spoken conversations with LLMs through browser interface. Complete speech-to-text (Whisper.cpp) and text-to-speech (Piper) pipeline with Ollama integration. Because why use existing solutions when you can build your own? \textit{Technologies: Java 21, Spring Boot, Ollama/LLaMA, WebSockets, Docker}

\textbf{SchulteDev Portfolio Repository} (2025): Centralized monorepo for professional portfolio including AI-powered CV generation, personal website, branding materials. Automated CV creation using Claude 4 with sophisticated GitHub Actions CI/CD pipeline. Professional PDF generation with automated releases. Meta-project for managing other projects. \textit{Technologies: Node.js, Jekyll, LaTeX, GitHub Actions, Claude 4 API}

\textbf{AI Document Processing for Tax Software} (Jan--May 2025): CLI application automating document capture using Azure Document Intelligence integrated with SteuerSparErklärung tax software. Because manually organizing tax documents is time-consuming, but automating it is somehow more time-consuming. \textit{Technologies: Golang, Azure Document Intelligence, SQLite}

\textbf{Atlassian Bamboo Server Add-ons} (Jun 2016--Dec 2023): Commercial Marketplace add-ons integrating code quality metrics (Checkstyle, PMD, CPD, FindBugs) into Bamboo CI/CD. Sustainable 7-year business with 83 customers across 15+ countries. First-mover advantage in Bamboo quality integration. Learned that niche markets are profitable until they're deprecated. \textit{Technologies: Java, Atlassian SDK, OSGi, jQuery, DataTables}

\textbf{Jenkins↔GitHub Integration PoC} (Apr--Jul 2018): Automatic Jenkins setup for GitHub repositories with webhooks and multi-tenant architecture. Proved concept nobody asked for. \textit{Technologies: Google Cloud, Firebase, Jenkins, Java, TypeScript, Golang}

\textbf{Crypto-Arbitrage Trading Bot} (Jan--Jul 2015): Automated Bitcoin arbitrage with machine learning. Worked in testing; markets moved faster than code in production. Expensive lesson in high-frequency trading realities. \textit{Technologies: Java, xChange API, Weka, LibSVM, MySQL}

\vfill

\newpage

% PAGE 4
\sectiontitle{Education (The Long Road)}

\textbf{Diplom in Computer Science (magna cum laude equivalent)} \hfill \textit{Sep 2012}\\
University of Koblenz-Landau, Campus Koblenz | Faculty of Computer Science

\textit{Focus:} Computational Visualistics and Software Engineering. The German Diplom is equivalent to Master of Science, awarded with distinction. \textit{Extended Duration:} Took longer than standard as I worked part-time/full-time as software developer starting 2007. Delayed studies but provided valuable practical experience---like learning that academic software engineering and production software engineering are different disciplines.

\vspace{6pt}

\textbf{Abitur (German high school diploma)} \hfill \textit{May 2003}\\
Gymnasium Haren (Ems), Germany | Advanced Courses: Mathematics and Physics

First contact with IT in sixth grade typing course. Last contact with PCs at school until graduation. Taught myself PHP (2000--2003) using PHP reference. Created homepages and forums for friend groups. Discovered computer games. Decided to study Computer Visualistics because games seemed fun. Learned that making games is less fun than playing them.

\vspace{8pt}

\sectiontitle{Career Breaks (Intentional and Otherwise)}

\textbf{Parental Leave} (Apr 2020--Sep 2021): COVID-19 pandemic lockdowns. Combined family time with industry-wide project slowdown affecting freelance opportunities. Learned that toddlers are more demanding than production outages but less predictable.

\textbf{Personal Travel Break} (Aug--Dec 2015): Extended travel exploring different countries and cultures. Intentional career pause. Returned refreshed and ready to make new mistakes.

\textbf{Military Service} (Aug 2003--Mar 2004): Mandatory German military service. Assigned to air force, completed basic training at Nassau-Dietz-Barracks (Budel, NL), transferred to Hopsten airbase (Rheine, DE). Learned discipline, following orders, and that military software is somehow worse than enterprise software.

\vspace{8pt}

\sectiontitle{Certifications \& Recognition}

\textbf{StackOverflow:} 4000+ reputation (\href{https://stackoverflow.com/users/1645517/markus-schulte}{profile}). Contributing answers since 2012. Discovered that explaining concepts to strangers improves understanding more than any course.

\textbf{Java SE 8 TechCheck (IKM):} 93/100 score, 88th percentile (Jun 2018). Overall assessment: ``Strong.'' 95\% proficient/strong knowledge across subjects. Validated that years of production Java actually taught something measurable.

\textbf{GitHub:} Open-source contributor (\href{https://github.com/SchulteDev}{profile}). Personal projects demonstrating that I willingly write code even when nobody's paying me.

\vspace{8pt}

\sectiontitle{Languages \& Additional Skills}

\textbf{Human Languages:} German (native), English (fluent/professional working proficiency)

\textbf{Soft Skills:} Technical mentoring (extensive experience making complex topics \textit{slightly} less confusing), stakeholder communication (translating ``the database is on fire'' into ``experiencing minor performance fluctuations''), agile coaching (facilitated countless retrospectives where we identified the same issues repeatedly), difficult conversations (removing team members, explaining technical debt to management, saying ``no'' to impossible deadlines).

\textbf{Community Contributions:} StackOverflow answers, open-source projects, technical blog posts nobody reads but I keep writing anyway.

\vspace{8pt}

\sectiontitle{Philosophy \& Approach}

Believe in applying IT best practices even when inconvenient. Skeptical of hype-driven development. Prefer boring, proven solutions over exciting, unproven ones (learned this the hard way). Value comprehensive testing, clear documentation, and automation---not because they're fun, but because 2 AM production outages are less fun.

Converted from microservices enthusiasm to Self-contained Systems pragmatism. Kubernetes skeptic. ElasticSearch advocate (eventually). Firm believer that simple solutions beat clever ones, though distinguishing ``simple'' from ``simplistic'' requires experience (i.e., mistakes).

\textit{``Crafting solutions that thrive today and tomorrow''} --- where ``tomorrow'' means ``when I'm not around to explain it.''

\vfill

\begin{center}
\small\textit{This Anti-CV was generated with assistance from Claude 4 (Anthropic AI) based on comprehensive career data.}\\
\textit{All failures, mistakes, and lessons learned are authentically mine. The AI just helped make them sound almost intentional.}\\
\textit{Document source: \href{https://github.com/SchulteDev/SchulteDev}{github.com/SchulteDev/SchulteDev}}
\end{center}

\end{document}