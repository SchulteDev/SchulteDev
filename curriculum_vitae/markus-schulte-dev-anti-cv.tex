\documentclass[10pt,a4paper]{article}
\usepackage[left=1.5cm,right=1.5cm,top=1.2cm,bottom=1.2cm]{geometry}
\usepackage{fontspec}
\usepackage{titlesec}
\usepackage{enumitem}
\usepackage{xcolor}
\usepackage{tabularx}
\usepackage{hyperref}
\usepackage{parskip}

\setmainfont{TeX Gyre Termes}
\setsansfont{TeX Gyre Heros}

\definecolor{failred}{RGB}{139,0,0}
\definecolor{lessonblue}{RGB}{25,25,112}
\definecolor{humorgray}{RGB}{80,80,80}

\titleformat{\section}{\large\bfseries\sffamily\color{failred}}{}{0em}{}[\titlerule]
\titleformat{\subsection}{\normalsize\bfseries\sffamily\color{lessonblue}}{}{0em}{}
\titlespacing*{\section}{0pt}{8pt}{4pt}
\titlespacing*{\subsection}{0pt}{6pt}{2pt}

\setlist[itemize]{nosep,leftmargin=1.2em,label={\textcolor{failred}{\textbullet}}}

\hypersetup{colorlinks=true,linkcolor=lessonblue,urlcolor=lessonblue}

\pagestyle{empty}

\begin{document}

\begin{center}
{\LARGE\bfseries\sffamily\color{failred} MARKUS SCHULTE}\\[2pt]
{\large\sffamily The Anti-CV: A Chronicle of Professional Humility}\\[4pt]
{\small\color{humorgray} Cologne, Germany \textbar{} \href{mailto:mail@schulte-development.de}{mail@schulte-development.de} \textbar{} +49 178 7217768}\\
{\small\color{humorgray} \href{https://schulte-development.de}{schulte-development.de} \textbar{} \href{https://linkedin.com/in/markus-schulte}{LinkedIn} \textbar{} \href{https://github.com/SchulteDev}{GitHub}}
\end{center}

\vspace{-4pt}

\section{Executive Summary of Inadequacy}

\textbf{Software Engineer \& Cloud Architect} with 19+ years of making mistakes professionally (12+ years freelancing through them). Currently serving as Cloud Architect at LBBW, where I've discovered that ``legacy application'' is just a polite term for ``someone else's regrets.'' My career has been a masterclass in learning things the hard way, choosing technologies that become obsolete, and discovering that the human factor in management is, in fact, a factor.

\textit{My Actual Slogan:} ``Building solutions that work despite my involvement''

\vspace{4pt}

\section{Greatest Hits: Career Disasters \& Hard-Won Wisdom}

\begin{tabularx}{\textwidth}{@{}p{3.2cm}X@{}}
\textbf{The Sphinx Incident} & Chose Sphinx over ElasticSearch in 2009 because it felt ``familiar.'' Watched the industry adopt ElasticSearch. Later migrated to ElasticSearch anyway. Time wasted: immeasurable.\\[4pt]
\textbf{Premature Leadership} & Accepted Head of Development role at age 24 with zero management experience. Discovered that ``technical competence'' and ``people skills'' are stored in different brain regions.\\[4pt]
\textbf{Microservices Reality} & Believed the microservices hype until Trusted Shops, where changing a button required PRs in 7 repositories. Now professionally skeptical.\\[4pt]
\textbf{Kubernetes Complexity} & Spent months learning Kubernetes at toom Baumarkt. Still believe it's a conspiracy by DevOps consultants to guarantee job security.\\[4pt]
\textbf{The Unnecessary Middleware} & Built an elegant Golang middleware at Saloodo that could have been replaced by 50 lines of AWS Lambda. Elegance: 10/10. Necessity: 2/10.\\
\end{tabularx}

\vspace{6pt}

\section{Skills I'm Still Learning (After 19 Years)}

\begin{tabularx}{\textwidth}{@{}p{3.5cm}X@{}}
\textbf{Languages} & Java (19y of ``why is this NullPointerException here?''), Golang (5y, still forgetting \texttt{if err != nil}), TypeScript (the type system judges me), PHP (8y, we don't talk about PHP)\\[3pt]
\textbf{Cloud Platforms} & Azure (discovering new ways to misconfigure), AWS (7y of accidentally leaving EC2 instances running), GCP (just enough to be dangerous)\\[3pt]
\textbf{Architecture} & Microservices (now reformed), Self-contained Systems (current obsession), Event-driven (because synchronous was too predictable)\\[3pt]
\textbf{Soft Skills} & Admitting when I'm wrong (ongoing), explaining technical concepts without condescension (feedback says ``needs work''), patience with legacy code (variable)\\[3pt]
\textbf{DevOps} & Terraform (infrastructure as code, bugs as infrastructure), Docker (it works on my machine), Kubernetes (see ``Disasters'' above)\\
\end{tabularx}

\vspace{6pt}

\section{Technologies That Have Humbled Me}

\begin{tabularx}{\textwidth}{@{}XXXX@{}}
\textbf{Databases} & \textbf{Testing} & \textbf{Monitoring} & \textbf{The Usual Suspects}\\
PostgreSQL, Oracle, MySQL (query optimization is an art I'm still learning) & JUnit, Pact, Playwright (1000+ tests, still find bugs in prod) & Grafana, Dynatrace, ELK (alerts at 3 AM build character) & Kafka, Redis, Terraform, Jenkins, too many to list\\
\end{tabularx}

\vfill

{\small\color{humorgray}\textit{``The expert has failed more times than the beginner has tried.'' --- Someone justifying my resume}}

\newpage

\section{Recent Roles: Where I Learned Humility}

\subsection{Cloud Architect at LBBW (Nov 2024 --- Present) | The Phoenix Project}

\textit{Discovering that ``legacy application'' is a term of endearment for software that has survived longer than most startups.}

\begin{itemize}
\item Cataloging hundreds of legacy applications for Azure migration; learned that every app has a story, and most stories end with ``the original developer left in 2015''
\item Applying R-methodology for migration assessment --- the ``R'' stands for ``Really, we're still running this?''
\item Evaluated Elastisys Welkin as a platform alternative; turns out evaluating platforms is easier than convincing people to change platforms
\item Creating Solution Design Templates that I hope will outlast my involvement (narrator: they might not)
\item Working on a platform called ``Skywalker'' that was production-unused during my engagement --- even Star Wars references don't guarantee adoption
\item Key realization: Banking IT moves at the speed of regulatory compliance, which is to say, deliberately
\end{itemize}

\textit{\small Technologies: Azure, AKS, Spring Boot, Tomcat, IIS, Oracle, MS-SQL, PostgreSQL, Jira, Confluence, endless Webex calls}

\vspace{4pt}

\subsection{Software Lead \& Cloud Architect at Union Investment (Oct 2021 --- Dec 2024)}

\textit{3+ years transforming a monolith into Self-contained Systems, one existential crisis at a time.}

\begin{itemize}
\item Built a microfrontend toolkit that I'm genuinely proud of --- which means I'll probably cringe at the code in 3 years
\item Made the executive decision to use Web Components without extensive PoC; saved 60 person-days, could have been a disaster
\item Chose Algolia over ElasticSearch --- learned from the Sphinx incident, finally made the less-familiar choice
\item Achieved 99.9\% uptime, which means we had exactly the right amount of downtime to keep us humble
\item Created 1000+ tests with 10-minute build times; still found a bug in production the week before go-live
\item Had to remove team members who weren't fitting despite integration attempts --- management lesson \#47: not every problem has a technical solution
\item 360\textdegree{} feedback noted I should ``consider pace of technical explanations for less technical stakeholders'' --- translation: I talk too fast about nerdy stuff
\end{itemize}

\textit{\small Technologies: Azure Static Web Apps, Functions, Event Hubs, TypeScript, Java, Lit, Algolia, Terraform, too much Nx configuration}

\vspace{4pt}

\subsection{Middleware Engineer at Saloodo!/DHL (Dec 2021 --- Nov 2024) | Part-Time Humility}

\textit{The project I took specifically to learn Golang professionally. Mission accomplished, lessons included.}

\begin{itemize}
\item Sole engineer for middleware syncing 10M records to Salesforce --- ``sole engineer'' is a fancy term for ``no one to blame but myself''
\item Achieved 100\% uptime over 3 years; the secret was comprehensive testing and living in fear
\item Built robust error handling for Saloodo's inconsistent database schema featuring malformed JSON and incorrect types --- my code has trust issues now
\item \textbf{The Big Lesson:} This entire middleware was architecturally unnecessary. AWS Lambda could have done this simpler, cheaper, faster. I built something elegant that didn't need to exist. Peak engineering.
\end{itemize}

\textit{\small Technologies: Golang, PostgreSQL, Salesforce Bulk API, Docker, ArgoCD, ELK Stack, Dynatrace, existential questioning}

\vspace{4pt}

\subsection{Cloud Consultant at Trusted Shops (Aug 2018 --- Oct 2019) | Microservices Disillusionment}

\textit{Where I learned that microservices are not a silver bullet, but rather a shotgun pointed at your deployment pipeline.}

\begin{itemize}
\item Migrated from JavaEE6 monolith to cloud-native architecture; the monolith was simpler, but we don't talk about that
\item Integrated Salesforce, Zuora, and Twilio --- three APIs, three different authentication schemes, three reasons for headaches
\item Led Pact contract testing workshops; turns out teaching people to write tests is harder than writing tests
\item \textbf{The Awakening:} Simple changes required modifications across many microservices. The IT architecture wasn't open to change. I became professionally skeptical of microservices and haven't recovered since.
\end{itemize}

\textit{\small Technologies: AWS Lambda, EKS, Spring Boot, Micronaut, Terraform, Pact, CloudWatch, Grafana, disillusionment}

\vfill

{\small\color{humorgray}\textit{``In theory, there's no difference between theory and practice. In practice, there is.'' --- Every architect ever}}

\newpage

\section{Earlier Career: Foundational Failures}

\subsection{Senior Engineer at toom Baumarkt (Oct 2016 --- Sep 2017) | The Kubernetes Incident}

\textit{``You build it, you run it'' sounds empowering until 3 AM.}

\begin{itemize}
\item First encounter with Kubernetes; spent more time debugging YAML indentation than writing business logic
\item Improved e2e test execution by 70\% --- mostly by deleting flaky tests (kidding, mostly)
\item Led cross-team quality initiatives across 4 teams; learned that ``quality'' means different things to different people
\item Reduced Java memory consumption through GC tuning; became briefly insufferable about garbage collection
\item \textbf{Lasting Opinion:} Kubernetes is a tool for platform teams at large organizations. For everyone else, it's a tool for generating consulting revenue. I remain skeptical.
\end{itemize}

\subsection{Earlier Freelance Projects (2014 --- 2016) | The Learning Curve}

\begin{itemize}
\item \textbf{YOOCHOOSE, Xsite, freshcells, fotocommunity, Silver Tours, AffiliCon, dailypresent} --- seven clients, seven different codebases, seven variations of ``we'll document this later''
\item Implemented CI/CD pipelines for 6+ companies; each one convinced me their setup was uniquely complex (it usually wasn't)
\item Established that recommending Jenkins would become a recurring theme in my career
\end{itemize}

\subsection{PHP Developer \& Head of Development at werkenntwen (2007 --- 2014)}

\textit{Germany's social network with 9M users. Yes, Germany had its own Facebook. No, it didn't survive.}

\begin{itemize}
\item Promoted to Head of Development at 24 with zero management experience; what could go wrong?
\item \textbf{Leadership Lesson:} Assumed technical competence equals leadership ability. Spent months focusing on architecture while team dynamics suffered. Eventually learned that humans are not microservices.
\item \textbf{The Sphinx Decision:} Chose Sphinx over ElasticSearch because it was familiar. Platform later migrated to ElasticSearch. I was right about needing search; wrong about which search.
\item Scaled architecture for 9M users; company still shut down in 2014 --- turns out technical excellence doesn't guarantee business survival
\item Introduced Jenkins CI in 2010; one decision I don't regret
\end{itemize}

\vspace{6pt}

\section{Personal Projects: Where I Fail on My Own Time}

\subsection{ConversationalAI4J (2025) | Current Obsession}

Java library for voice-enabled AI using local models. Because talking to my computer wasn't weird enough, now I'm building the infrastructure for it. Uses Whisper.cpp, Piper TTS, and Ollama. Works approximately 73\% of the time.

\subsection{AI Document Processing for Tax Software (2025)}

CLI tool integrating Azure Document Intelligence with German tax software. The German government's document requirements met Azure's AI. Both were confused, but it works.

\subsection{Crypto-Arbitrage Trading Bot (2015) | The One That Got Away}

Built Bitcoin arbitrage algorithms with machine learning. Made some money, learned a lot about market timing, ultimately decided that my risk tolerance is better suited to enterprise software.

\vspace{6pt}

\section{Technologies I've Survived}

\begin{tabularx}{\textwidth}{@{}p{2.2cm}X@{}}
\textbf{2007--2014} & PHP, MySQL, Zend Framework, Jenkins, ``Web 2.0'' (remember that term?)\\
\textbf{2014--2018} & AWS, Docker, Microservices, OAuth2, Kubernetes, the containerization hype\\
\textbf{2019--Now} & Azure, Golang, Event-driven architecture, Self-contained Systems, AI integration, accepting that everything is a distributed system\\
\end{tabularx}

\vfill

{\small\color{humorgray}\textit{``The best code is the code you don't have to write.'' --- Me, after building unnecessary middleware}}

\newpage

\section{Education: The Extended Director's Cut}

\subsection{Diplom in Computer Science --- University of Koblenz-Landau (2004 --- 2012)}

\begin{itemize}
\item \textbf{Duration:} 8 years for a degree designed for 5 --- efficiency was not my strong suit
\item \textbf{Excuse:} Worked part-time (sometimes full-time) as a developer starting in 2007; gained practical experience while theoretical knowledge gathered dust
\item \textbf{Result:} Graduated magna cum laude despite everything; apparently showing up eventually works
\item \textbf{Focus:} Computational Visualistics and Software Engineering --- I wanted to make video games, ended up in enterprise software
\item \textbf{Honest Assessment:} The extended timeline taught me that deadlines are negotiable and that real-world experience often outweighs theoretical knowledge (don't tell academia)
\end{itemize}

\subsection{Abitur --- Gymnasium Haren (2003)}

Advanced courses in Mathematics and Physics. Taught myself PHP from a reference book in 2000--2003. Built forums for friends. The path to enterprise architecture was not obvious.

\vspace{6pt}

\section{Certifications \& Proof I Occasionally Know Things}

\begin{itemize}
\item \textbf{Java SE 8 TechCheck (IKM):} Scored 93/100, 88th percentile. Still forget what \texttt{transient} does.
\item \textbf{StackOverflow:} 4000+ reputation. Mostly from answering questions I previously googled myself.
\item \textbf{GitHub:} Active contributor. My commit messages have improved from ``fix'' to ``fix: actually fix the thing this time.''
\end{itemize}

\vspace{6pt}

\section{Career Breaks: Strategic Retreats}

\begin{tabularx}{\textwidth}{@{}p{3cm}X@{}}
\textbf{Apr 2020 --- Sep 2021} & \textbf{Parental Leave} during COVID-19. Discovered that debugging a toddler has no Stack Overflow answers.\\[3pt]
\textbf{Aug --- Dec 2015} & \textbf{Travel Break.} Explored countries. Returned with perspective and a depleted savings account.\\[3pt]
\textbf{Aug 2003 --- Mar 2004} & \textbf{Mandatory Military Service.} German Air Force. Learned that large organizations move slowly regardless of industry.\\
\end{tabularx}

\vspace{6pt}

\section{What I've Actually Learned (The Serious Bit)}

\begin{itemize}
\item \textbf{Quality pays for itself:} High testing standards at Saloodo delivered speed improvements that quickly repaid the investment
\item \textbf{Bias for action:} Sometimes a well-founded decision beats analysis paralysis (saved 60 person-days at Union Investment)
\item \textbf{Architecture evolves:} The ``right'' choice depends on context; what was right in 2009 (Sphinx) was wrong by 2014
\item \textbf{People matter more than code:} My biggest growth has been in leadership and communication, not technical skills
\item \textbf{Simplicity wins:} The best solution is often the boring one (SCS over microservices, managed services over Kubernetes)
\item \textbf{Stay humble:} 19 years in, I'm still learning. That's either concerning or encouraging, depending on perspective.
\end{itemize}

\vspace{4pt}

\section{Languages}

\textbf{German:} Native (including the ability to form compound words of arbitrary length)\\
\textbf{English:} Fluent (professional proficiency, occasional creative grammar)

\vfill

\begin{center}
\rule{0.6\textwidth}{0.4pt}\\[4pt]
{\footnotesize\color{humorgray} \textbf{AI Transparency Notice:} This Anti-CV was collaboratively created with AI assistance (Claude), because even self-deprecating humor benefits from editorial review. All failures, lessons, and embarrassing career moments are authentically mine. The AI just helped me laugh at them more eloquently.}\\[6pt]
{\footnotesize\color{humorgray} \textit{``The only real mistake is the one from which we learn nothing.'' --- Henry Ford}\\
\textit{Fortunately, I've learned something from most of mine.}}
\end{center}

\end{document}