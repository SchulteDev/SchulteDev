\documentclass[11pt,a4paper]{article}
\usepackage[utf8]{inputenc}
\usepackage[T1]{fontenc}
\usepackage{lmodern}
\usepackage[margin=0.7in]{geometry}
\usepackage{enumitem}
\usepackage{titlesec}
\usepackage{xcolor}
\usepackage{hyperref}
\usepackage{tabularx}
\usepackage{array}

\definecolor{darkblue}{RGB}{0,50,100}
\definecolor{lightgray}{RGB}{128,128,128}

\hypersetup{
    colorlinks=true,
    linkcolor=darkblue,
    urlcolor=darkblue,
    citecolor=darkblue
}

\titleformat{\section}{\Large\bfseries\color{darkblue}}{}{0em}{}[\titlerule]
\titleformat{\subsection}{\large\bfseries\color{darkblue}}{}{0em}{}

\setlength{\parindent}{0pt}
\setlength{\parskip}{0.5em}

\pagestyle{empty}

\begin{document}

\begin{center}
{\Huge \textbf{Markus Schulte}}\\[0.3em]
{\Large \textbf{Anti-CV: A Humorous Journey Through 18 Years of ``Learning Experiences''}}\\[0.3em]
{\large Freelance Cloud Architect \& Professional Mistake Collector}\\[0.3em]
\href{mailto:mail@schulte-development.de}{mail@schulte-development.de} | +49 178 7217768 | Cologne, Germany\\
\href{https://schulte-development.de}{schulte-development.de} | \href{https://github.com/SchulteDev}{GitHub} | \href{https://linkedin.com/in/markus-schulte}{LinkedIn}
\end{center}

\vspace{0.5em}

\section*{Professional Summary: 18 Years of ``Character Building''}

\textbf{Cloud Architect \& Software Engineer} who has spent 18+ years learning why things don't work before figuring out how they should. Currently helping LBBW's ``Phoenix'' rise from the ashes of legacy systems, having previously survived the trenches of enterprise modernization at Union Investment.

\textbf{My Journey of Educational Failures:}
\begin{itemize}[nosep,leftmargin=1.5em]
    \item \textbf{Management Hubris:} Accepted Head of Development role at age 25 without management experience. Learned that technical skills don't magically translate to people skills. Evolved from ``dictator'' to ``laissez-faire'' leadership through trial and error.
    \item \textbf{Technology FOMO:} Chose Sphinx over ElasticSearch in 2010 because ES seemed ``too exotic.'' Platform later migrated to ElasticSearch anyway. Lesson: Sometimes the shiny new thing is actually better.
    \item \textbf{Microservices Reality Check:} First encounter with ``real'' microservices at Trusted Shops revealed the lived disadvantages nobody talks about. Simple changes required touching 5+ services. Now prefer Self-contained Systems after learning the hard way.
    \item \textbf{Kubernetes Skepticism:} Discovered Kubernetes at toom Baumarkt. Concluded it's too complex for mere mortals. Still believe it's a tool for ``IT operations solution providers,'' not regular humans trying to deploy apps.
    \item \textbf{Solo Developer Syndrome:} Spent 3 years as the only developer maintaining Saloodo's middleware. Achieved 100\% uptime through paranoia and excessive testing. Learned that being irreplaceable is both a blessing and a curse.
\end{itemize}

\textbf{Accidental Wins That Somehow Worked:}
\begin{itemize}[nosep,leftmargin=1.5em]
    \item \textbf{Bias for Action:} Saved 60 person-days at Union Investment by implementing Web Components without exploring alternatives. Sometimes decisive action beats analysis paralysis.
    \item \textbf{Quality Obsession:} Implemented comprehensive testing everywhere, even when colleagues thought it was overkill. Turns out 1000+ tests and 10-minute build cycles are possible and profitable.
    \item \textbf{Technology Pragmatism:} Chose Algolia over ElasticSearch after thorough analysis. Sometimes the ``easy'' solution is the right solution.
\end{itemize}

\vspace{0.5em}

\section*{Core ``Expertise'' \& Hard-Won Lessons}

\begin{tabularx}{\textwidth}{>{\bfseries}p{4cm}X}
\textbf{Languages} & Java (18y of Stockholm syndrome), Golang (4y of enlightenment), PHP (8y of ``how did this work?''), TypeScript (as needed for survival) \\
\textbf{Cloud Platforms} & AWS (7y of ``it's always DNS''), Azure (4y of ``why is everything called 'App'?''), GCP (occasional flirtation) \\
\textbf{Architecture} & Microservices (skeptical), Self-contained Systems (convert), Event-driven (when it works), Monoliths (honest about trade-offs) \\
\textbf{DevOps} & Docker (useful), Kubernetes (too complex), Terraform (love-hate relationship), CI/CD (essential for sanity) \\
\textbf{Leadership} & Team building (learned through mistakes), Stakeholder management (ongoing education), Product Owner (interim survival) \\
\textbf{Testing} & Unit testing (religious), Integration testing (practical), Contract testing (Pact evangelist), E2E testing (necessary evil) \\
\textbf{Databases} & PostgreSQL (reliable friend), MySQL (old companion), Oracle (expensive acquaintance), MS-SQL (when required) \\
\end{tabularx}

\vspace{0.5em}

\section*{Business Impact: When Things Actually Worked}

\begin{itemize}[nosep,leftmargin=1.5em]
    \item \textbf{Platform Scale:} Delivered systems serving 9M+ users (wer-kennt-wen.de) - before learning that scale brings unique problems
    \item \textbf{Deployment Speed:} Reduced deployment time from days to 30 minutes - after experiencing the pain of manual deployments
    \item \textbf{System Reliability:} Achieved 99.9\% uptime through paranoid testing and monitoring - learned from production failures
    \item \textbf{Team Productivity:} Enabled 10x faster deployment cycles through automation - after manually deploying for years
    \item \textbf{Architecture Templates:} Created reusable patterns adopted across enterprise - after seeing too many reinvented wheels
\end{itemize}

\textbf{My Slogan:} ``Crafting solutions that thrive today and tomorrow'' (after making enough mistakes to know what doesn't work)

\newpage

\section*{Recent Professional Experience: The Learning Continues}

\subsection*{November 2024 --- Present: Cloud Architect at LBBW}
\textit{Landesbank Baden-Württemberg | ``Phoenix'' IT Modernization}

\begin{itemize}[nosep,leftmargin=1.5em]
    \item \textbf{Legacy Assessment Reality:} Cataloging hundreds of applications for Azure migration using R-methodology. Discovered that ``legacy'' is a diplomatic term for ``nobody knows how this works anymore.''
    \item \textbf{Platform Gap Analysis:} Evaluating production-unused Skywalker platform. Learning that ``cloud-ready'' and ``production-ready'' are different things entirely.
    \item \textbf{Banking Compliance:} Navigating complex regulatory requirements while modernizing. Discovering that ``move fast and break things'' doesn't work in banking.
    \item \textbf{Stakeholder Translation:} Explaining technical concepts to business committees. Perfecting the art of saying ``it's complicated'' in professional language.
\end{itemize}

\subsection*{October 2021 --- December 2024: Cloud Architect at Union Investment}
\textit{3+ Year B2B Platform Modernization Journey}

\begin{itemize}[nosep,leftmargin=1.5em]
    \item \textbf{Monolith Decomposition:} Broke apart overloaded Liferay/FirstSpirit system into 8 Self-contained Systems. Learned that Domain-Driven Design is harder than the books make it look.
    \item \textbf{Team Building Adventures:} Built teams from scratch, including the difficult decision to remove members who didn't fit. Discovered that ``culture fit'' isn't just HR speak.
    \item \textbf{Microfrontend Framework:} Created toolkit enabling autonomous team development. Chose Web Components over trendy frameworks --- sometimes boring technology is the right choice.
    \item \textbf{Search Architecture:} Designed unified search across distributed systems using Azure Event Hubs and Algolia. Learned that ``impossible'' problems often have simple solutions.
\end{itemize}

\subsection*{December 2021 --- November 2024: Middleware Engineer at Saloodo! (DHL)}
\textit{Part-time Golang Adventure}

\begin{itemize}[nosep,leftmargin=1.5em]
    \item \textbf{Solo Developer Life:} Sole maintainer of critical Salesforce sync middleware. Achieved 100\% uptime through paranoid testing and comprehensive monitoring.
    \item \textbf{Database Horror Stories:} Handled ~10M records with malformed JSON and incorrect types. Learned that robust error handling is not optional.
    \item \textbf{Golang Learning:} First professional Golang project. Discovered that simple languages can solve complex problems elegantly.
    \item \textbf{Architecture Skepticism:} Realized the middleware was architecturally unnecessary. Sometimes the best solution is the one you don't build.
\end{itemize}

\subsection*{August 2018 --- October 2019: Cloud Consultant at Trusted Shops}
\textit{Microservices Reality Check}

\begin{itemize}[nosep,leftmargin=1.5em]
    \item \textbf{Microservices Awakening:} First encounter with ``real'' microservices architecture. Surprised by lived disadvantages --- simple changes required touching many services.
    \item \textbf{AWS Lambda Optimization:} Created Java Lambda templates with Micronaut. Learned that framework choice matters more in serverless environments.
    \item \textbf{Contract Testing Evangelism:} Introduced Pact testing across multiple teams. Discovered that preventing integration issues is easier than fixing them.
    \item \textbf{Infrastructure as Code:} Implemented Terraform for AWS resources. Learned that ``infrastructure drift'' is a real problem with real consequences.
\end{itemize}

\vspace{0.5em}

\section*{Technology Evolution: A Timeline of Trends Survived}

\textbf{2007-2014: The Foundation Years}
\begin{itemize}[nosep,leftmargin=1.5em]
    \item Started with PHP and MySQL. Learned object-oriented programming the hard way by refactoring procedural spaghetti code.
    \item Introduced CI/CD with Jenkins. Convinced skeptical teams that automated testing wasn't just for ``big companies.''
\end{itemize}

\textbf{2014-2018: Cloud Native Awakening}
\begin{itemize}[nosep,leftmargin=1.5em]
    \item Embraced Docker containers. Discovered that ``works on my machine'' is not a deployment strategy.
    \item Explored Kubernetes at toom Baumarkt. Concluded it's too complex for most use cases. Still maintain this ``unpopular'' opinion.
\end{itemize}

\textbf{2019-Present: Enterprise Reality}
\begin{itemize}[nosep,leftmargin=1.5em]
    \item Mastered Azure services. Learned that every cloud has its own special way of doing the same things.
    \item Adopted Self-contained Systems over microservices. Sometimes the less trendy approach is more practical.
\end{itemize}

\newpage

\section*{Earlier Career: Where It All Began}

\subsection*{November 2017 --- March 2018: Senior JavaEE Developer at Allianz}
\textit{via SinnerSchrader | CloudFoundry Platform Experience}

Brief but educational stint developing JavaEE applications for Allianz digital initiatives. Learned that enterprise development involves more meetings than expected. Successfully deployed applications to CloudFoundry (AWS-hosted), discovering that platform-as-a-service can actually work when properly implemented.

\subsection*{October 2016 --- September 2017: Senior Engineer at toom Baumarkt}
\textit{via tarent solutions | Microservices \& Kubernetes Introduction}

Worked in 8-member ``Butterkekse'' (butter biscuit) Scrum team following ``you build it, you run it'' principle. Led quality initiatives across 4 development teams. First encounter with Kubernetes --- concluded it was overly complex for individual applications. Developed production microservices in Java and Golang, reducing deployment time from days to hours through automation.

\subsection*{August 2007 --- June 2014: PHP Developer \& Head of Development at werkenntwen}
\textit{Germany's Social Network | 9+ Million Members}

\textbf{Career Progression:} PHP Developer → Head of Development (age 25) → Backend Developer

\textbf{Key Learning Experience:} Accepted Head of Development role without management experience. Initially focused on technical aspects, underestimating human factors. Through training and reflection, evolved toward laissez-faire leadership style. Led platform scaling for 9+ million users, implementing CI/CD, comprehensive testing, and performance optimization.

\textbf{Technology Decision Regret:} Chose Sphinx over ElasticSearch in 2010 because ES seemed ``too exotic.'' Platform later migrated to ElasticSearch anyway. Lesson learned: sometimes the innovative choice is the right choice.

\vspace{0.5em}

\section*{Freelance Portfolio (2014-2016) \& Product Business}

\textbf{Early Freelancing:} Multiple short-term contracts including YOOCHOOSE (AWS operations), Xsite (REST APIs), freshcells (Docker CI), fotocommunity (Bamboo CI), Silver Tours (infrastructure modernization). Learned that consulting requires different skills than development.

\textbf{Atlassian Marketplace Success (2016-2023):} Created Java-based add-ons for Bamboo Server integrating code quality metrics. Built sustainable business with 83 customers across 15+ countries. Discovered that solving niche problems can be profitable. Maintained 7-year product lifecycle until market obsolescence.

\vspace{0.5em}

\section*{Personal Projects: Where I Experiment}

\textbf{2025: SchulteDev Portfolio Repository}
Centralized monorepo with AI-powered CV generation using Claude 4. Automated LaTeX document generation with GitHub Actions. Sometimes overengineering personal projects teaches valuable lessons about automation.

\textbf{2025: AI Document Processing for Tax Software}
Golang CLI application using Azure Document Intelligence for automated document capture. Learned that AI can solve real-world problems when properly applied.

\textbf{2018: Jenkins ↔ GitHub Integration PoC}
Built integration for automatic Jenkins setup with webhooks. Discovered that automation can solve setup complexity, but maintenance is still required.

\textbf{2015: Crypto-Arbitrage Trading Bot}
Java-based Bitcoin trading algorithms with machine learning. Learned that markets are efficient and automated trading is harder than it looks.

\vspace{0.5em}

\section*{Recognition \& Community}

\textbf{StackOverflow:} 4000+ reputation points from helping others avoid my mistakes\\
\textbf{Java SE 8 TechCheck:} 93/100 score (88th percentile) --- proof that experience eventually pays off\\
\textbf{GitHub:} Open-source contributions and personal projects demonstrating ongoing learning

\newpage

\section*{Education: The Theoretical Foundation}

\subsection*{2004 --- 2012: Diploma in Computer Science}
\textit{University of Koblenz-Landau | Computational Visualistics Focus}

\textbf{Degree:} Diploma in Computer Science (equivalent to Master's) with distinction\\
\textbf{Extended Timeline:} Program took longer than standard duration because I worked part-time (and sometimes full-time) as a software developer starting in 2007. Learned that practical experience and academic learning complement each other well.

\textbf{Key Insight:} University taught me theoretical foundations, but real-world software development taught me why those foundations matter. The combination of academic knowledge and practical experience proved invaluable.

\subsection*{2003: Abitur (German High School Diploma)}
\textit{Gymnasium Haren (Ems) | Advanced Courses: Mathematics \& Physics}

Early contact with IT: Taught myself PHP using reference books (2000-2003). Created homepages and forums for friends and graduating class. Discovered that programming was more fun than physics homework.

\vspace{0.5em}

\section*{Additional Skills: The Broader Toolkit}

\textbf{Languages:} German (native), English (fluent professional)\\
\textbf{Methodologies:} Agile/Scrum (lived experience), SAFe (survived), Kanban (practical)\\
\textbf{Architecture Patterns:} Clean Architecture, Domain-Driven Design, Event-driven Architecture\\
\textbf{Leadership:} Technical mentoring, team building, stakeholder management, Product Owner experience\\
\textbf{Quality Practices:} Test-driven development, continuous integration, code review culture\\
\textbf{Documentation:} Technical writing, API documentation, architecture decision records

\vspace{0.5em}

\section*{Career Breaks: When Life Intervenes}

\textbf{April 2020 --- September 2021: Parental Leave}
Combined family time with industry-wide project slowdown during COVID-19 pandemic. Learned that stepping away from technology occasionally provides valuable perspective.

\textbf{August 2015 --- December 2015: Personal Travel Break}
Extended travel period exploring different countries and cultures. Discovered that problem-solving skills translate across cultures, but communication styles vary significantly.

\textbf{August 2003 --- March 2004: Military Service}
Mandatory German military service in air force. Completed basic training in Netherlands, then stationed at Hopsten airbase. Learned that hierarchical organizations have different challenges than flat development teams.

\vspace{0.5em}

\section*{Philosophy: What I've Learned}

\textbf{On Technology:} Every technology is a tool with trade-offs. Popularity doesn't equal suitability. Sometimes the boring solution is the right solution.

\textbf{On Architecture:} Simple systems are easier to understand, maintain, and debug. Complexity should be justified by actual requirements, not theoretical possibilities.

\textbf{On Teams:} Good people are more important than perfect processes. Clear communication prevents more problems than sophisticated tools.

\textbf{On Quality:} Comprehensive testing and automation pay for themselves quickly. Technical debt is like financial debt --- it compounds over time.

\textbf{On Leadership:} Management is a skill that must be learned, not an automatic promotion. Leading by example is more effective than leading by authority.

\vspace{0.5em}

\section*{Current Interests \& Future Learning}

\textbf{AI Integration:} Exploring practical applications of AI in software development and business processes\\
\textbf{Platform Engineering:} Understanding how to build developer platforms that teams actually want to use\\
\textbf{Sustainability:} Investigating how software architecture decisions impact environmental footprint\\
\textbf{Distributed Systems:} Continuing to learn about eventual consistency and distributed system trade-offs

\vfill

\begin{center}
\textcolor{lightgray}{\small This Anti-CV was generated with assistance from Anthropic's Claude 4 AI model, based on comprehensive career data. The failures and lessons are authentically mine; the humor was enhanced by AI. Because even CV generation can be improved with automation.}
\end{center}

\end{document}