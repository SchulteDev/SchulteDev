\documentclass[11pt,a4paper]{article}
\usepackage{fontspec}
\usepackage{geometry}
\usepackage{graphicx}
\usepackage{xcolor}
\usepackage{enumitem}
\usepackage{titlesec}
\usepackage{tabularx}
\usepackage{array}
\usepackage{hyperref}
\usepackage{fancyhdr}
\usepackage{lastpage}
\usepackage{microtype}

\geometry{margin=0.7in}
\setmainfont{Latin Modern Roman}
\setsansfont{Latin Modern Sans}
\setmonofont{Latin Modern Mono}

\definecolor{darkblue}{RGB}{0,50,100}
\definecolor{lightgray}{RGB}{240,240,240}
\definecolor{mediumgray}{RGB}{128,128,128}

\hypersetup{
    colorlinks=true,
    linkcolor=darkblue,
    urlcolor=darkblue,
    pdfauthor={Markus Schulte},
    pdftitle={Anti-CV: Lessons from Professional Failures}
}

\pagestyle{fancy}
\fancyhf{}
\fancyhead[L]{\textcolor{mediumgray}{\small Anti-CV: Markus Schulte}}
\fancyhead[R]{\textcolor{mediumgray}{\small Page \thepage\ of \pageref{LastPage}}}
\renewcommand{\headrulewidth}{0.4pt}
\renewcommand{\headrule}{\hbox to\headwidth{\color{lightgray}\leaders\hrule height \headrulewidth\hfill}}

\titleformat{\section}{\Large\bfseries\sffamily\color{darkblue}}{}{0em}{}[\vspace{-0.5em}\textcolor{lightgray}{\hrulefill}]
\titleformat{\subsection}{\large\bfseries\sffamily}{}{0em}{}
\titleformat{\subsubsection}{\normalsize\bfseries\sffamily}{}{0em}{}

\setlength{\parindent}{0pt}
\setlength{\parskip}{0.3em}
\setitemize{leftmargin=1em,itemsep=0.1em,parsep=0em}

\begin{document}

% Page 1: Header, Philosophy, Key Failures Overview
\begin{center}
{\Huge\bfseries\sffamily\color{darkblue} ANTI-CV}\\[0.3em]
{\Large\sffamily Professional Failures \& Hard-Learned Lessons}\\[0.5em]
{\large\bfseries Markus Schulte}\\[0.2em]
{\normalsize Freelance Cloud Architect \& Software Engineer}\\[0.2em]
{\small Cologne, Germany $\cdot$ \href{mailto:mail@schulte-development.de}{mail@schulte-development.de} $\cdot$ +49 178 7217768}\\[0.2em]
{\small \href{https://schulte-development.de}{schulte-development.de} $\cdot$ \href{https://github.com/SchulteDev}{GitHub} $\cdot$ \href{https://linkedin.com/in/markus-schulte}{LinkedIn}}
\end{center}

\vspace{0.5em}

\section{Why This Anti-CV Exists}

After 18+ years in software development, I've learned that failures teach more than successes. This document celebrates the spectacular mistakes, wrong turns, and face-palm moments that shaped my career. Because nothing says ``experienced developer'' like a collection of expensive lessons learned the hard way.

\textbf{The Philosophy:} Traditional CVs hide the messy reality of professional growth. This anti-CV embraces it, showing how wrong decisions, over-confidence, and plain old mistakes led to better judgment, humility, and surprisingly, better opportunities.

\section{Greatest Hits: My Top Professional Failures}

\subsection{The Premature Promotion Disaster (2008)}
\textbf{The Setup:} Got promoted to Head of Development at werkenntwen after just 1 year as a junior PHP developer. Company logic: ``He's technical, how hard can management be?''

\textbf{The Reality Check:} Turns out, knowing how to code doesn't automatically make you good at managing humans. Who knew? Spent months trying to solve people problems with technical solutions. Attempted to implement ``performance metrics'' for developers like they were database queries.

\textbf{The Lesson:} Management is a different skill set entirely. Technical competence $\neq$ leadership ability. Eventually evolved into a laissez-faire leadership style that actually worked, but only after some painful learning experiences.

\textbf{The Redemption:} Later successfully led teams of 5-9 developers at Union Investment, applying hard-won people skills alongside technical expertise.

\subsection{The Conservative Technology Choice Catastrophe (2009)}
\textbf{The Disaster:} Chose Sphinx search engine over ElasticSearch because ``ElasticSearch looked too exotic and complicated.'' Played it safe with the familiar option.

\textbf{The Irony:} Two years later, the platform migrated from Sphinx to ElasticSearch anyway, proving that the ``exotic'' choice was actually the future. My conservative decision cost the company a migration project.

\textbf{The Lesson:} Sometimes the scary new technology is scary because it's better. Conservative choices aren't always the right choices.

\subsection{The Kubernetes Complexity Shock (2016)}
\textbf{The Hype:} ``Kubernetes is the future! Container orchestration! Cloud native!''

\textbf{The Reality:} Spent months at toom Baumarkt wrestling with Kubernetes deployments. Realized that this ``simple'' container orchestration required an entire ops team just to babysit YAML files.

\textbf{The Honest Take:} Still believe Kubernetes is overly complex for most applications. Great for platform companies, overkill for business applications. This opinion makes me unpopular at cloud conferences.

\textbf{The Practical Outcome:} Now recommend managed services over Kubernetes for most clients. Surprisingly, this ``controversial'' stance has saved clients significant time and money.

\subsection{The Microservices Disillusionment (2018)}
\textbf{The Promise:} Joined Trusted Shops excited to work with ``real microservices architecture.'' Expected the scalability and maintainability benefits I'd read about.

\textbf{The Harsh Reality:} Simple changes required modifications across 6+ microservices. Lack of proper testing and automation made the distributed system a maintenance nightmare. Spent more time debugging service interactions than building features.

\textbf{The Revelation:} Microservices architecture can be worse than monoliths without proper tooling, testing, and team discipline. The distributed complexity isn't always worth the theoretical benefits.

\textbf{The Evolution:} Now advocate for Self-contained Systems (SCS) as implemented at Union Investment -- the benefits of service separation without the operational complexity.

\section{Professional Achievements (Despite the Failures)}

\textbf{Current Role:} Cloud Architect at LBBW (Germany's major state bank) -- somehow they trusted me with their ``Phoenix'' IT modernization project involving 80+ people and hundreds of legacy applications.

\textbf{Recent Success:} Led 3-year cloud transformation at Union Investment, serving as Cloud Architect for platform used by 80,000+ users. Established Self-contained Systems architecture as corporate standard. Reduced deployment times from days to 30 minutes.

\textbf{The Irony:} My failures with microservices and Kubernetes directly informed the successful SCS architecture at Union Investment. Previous disasters became architectural wisdom.

\textbf{Technical Expertise:} Java (18 years), Golang (4 years), AWS/Azure/GCP, event-driven architecture, contract testing evangelism. Still occasionally write PHP when nobody's looking.

\textbf{Leadership Evolution:} Successfully managed teams of 5-9 developers after learning that people aren't deployable code units. Who would have thought?

\vspace{0.5em}

\begin{tabularx}{\textwidth}{Xr}
\textbf{Technologies I Actually Use Well:} & \textbf{Technologies I Avoid:} \\
Java/Spring Boot, Golang, TypeScript & Kubernetes (unless absolutely necessary) \\
AWS/Azure (managed services) & Microservices (without proper tooling) \\
Self-contained Systems, Event-driven architecture & Overly complex solutions \\
Contract testing (Pact), Docker & Hype-driven development \\
Terraform, GitHub Actions & Technologies I choose for resume appeal \\
\end{tabularx}

\newpage

% Page 2: Recent Major Roles with Failure Focus
\section{Recent Professional Adventures (2021-Present)}

\subsection{Cloud Architect at LBBW (November 2024 -- Present)}
\textbf{The Challenge:} Modernizing a major German state bank's IT infrastructure. ``Phoenix'' project migrating hundreds of legacy applications to Azure ``Skywalker'' platform.

\textbf{The Complexity:} Cataloging applications that range from Spring Boot to systems older than some of my colleagues. Each application is a special snowflake requiring individualized migration approach.

\textbf{The Reality Check:} Working with a production-unused platform that needs enhancement while simultaneously planning migrations. It's like renovating a house while the blueprints are still being drawn.

\textbf{The Honest Assessment:} Banking regulations make everything 3x more complex than necessary. But the methodical approach to legacy application assessment is oddly satisfying -- like archaeology, but with more YAML.

\textbf{Key Contributions:} Applied R-methodology for cloud migration assessment, developed technical templates for systematic application evaluation, conducted Azure cost analysis that probably scared some executives.

\subsection{Cloud Architect at Union Investment (October 2021 -- December 2024)}
\textbf{The Setup:} 3-year modernization project for Germany's leading asset management company. Transform legacy Liferay monolith into modern Self-contained Systems architecture.

\textbf{The Plot Twist:} Despite serving only 80,000 users, this platform was business-critical. No pressure.

\textbf{The Architectural Journey:}
\begin{itemize}
\item \textbf{Phase 1:} Designed microfrontend architecture. Spent months debating Web Components vs. React vs. Angular. Chose Web Components because standards matter (and because I'm stubborn).
\item \textbf{Phase 2:} Built development team from scratch. Turns out hiring good developers is harder than writing good code.
\item \textbf{Phase 3:} Implemented event-driven search across 8 domains. Chose Algolia over ElasticSearch because sometimes the expensive option is worth it.
\end{itemize}

\textbf{The Lessons:}
\begin{itemize}
\item Self-contained Systems work better than microservices for business applications
\item Web Components are underrated (probably because they're not backed by big tech companies)
\item Event-driven architecture is powerful but requires discipline
\item Azure Static Web Apps are surprisingly good for microfrontends
\end{itemize}

\textbf{The Success Metrics:} 
\begin{itemize}
\item Deployment time: Days $\rightarrow$ 30 minutes
\item 99.9\% uptime in production
\item 1000+ tests running in 10 minutes (parallelization works)
\item Team satisfaction: Multiple 360-degree feedback surveys rating technical competence 9-10/10
\end{itemize}

\textbf{The Honest Feedback:} Team members noted I sometimes explain things too technically for non-technical stakeholders. Working on translating ``event-driven architecture'' into business language.

\subsection{Middleware Engineer at Saloodo!/DHL (December 2021 -- November 2024)}
\textbf{The Setup:} Part-time Golang project while working at Union Investment. Synchronizing 10+ million records between Saloodo platform and Salesforce.

\textbf{The Challenge:} Saloodo's database schema was... creative. Incorrect types, malformed JSON, and data inconsistencies that would make a DBA cry.

\textbf{The Solution:} Built fault-tolerant middleware that could handle Saloodo's ``unique'' data quality while maintaining 100\% uptime for 3 years.

\textbf{The Honest Truth:} Sometimes you have to work with systems that make you question your career choices. But reliable software that handles bad data gracefully is oddly satisfying to build.

\textbf{The Technical Win:} First professional Golang project. Achieved 70\% test coverage on 2,776 lines of code with 5-second test execution. Turns out Golang is actually nice to work with.

\section{The Wisdom of Earlier Failures (2016-2020)}

\subsection{The Atlassian Marketplace Adventure (2016-2023)}
\textbf{The Business Idea:} Build commercial add-ons for Atlassian Bamboo Server. How hard could product development be?

\textbf{The Reality:} Harder than expected. Dealing with customers, licensing, support, and the eventual death of Atlassian Server products.

\textbf{The Outcome:} 83 customers across 15+ countries, 7 years of sustainable revenue. Not bad for a side project that started as ``let's see what happens.''

\textbf{The Lesson:} Sometimes accidental businesses work out. Also, Atlassian's decision to kill Server products taught me about platform risk.

\subsection{The Trusted Shops Microservices Reality Check (2018-2019)}
\textbf{The Expectation:} Modern microservices architecture at a successful e-commerce company. Clean code, proper testing, scalable systems.

\textbf{The Shock:} Microservices without proper testing and automation are worse than monoliths. Simple changes cascaded across multiple services.

\textbf{The Contribution:} Introduced contract testing (Pact) and improved Clean Code practices across 13+ repositories. Applied Infrastructure-as-Code principles.

\textbf{The Lasting Impact:} This experience directly influenced the SCS architecture decisions at Union Investment. Sometimes you have to see what doesn't work to design what does.

\subsection{The toom Baumarkt DevOps Learning Curve (2016-2017)}
\textbf{The Role:} Senior Engineer in ``you build it, you run it'' microservices team.

\textbf{The Docker/Kubernetes Initiation:} First encounter with containerization. Kubernetes seemed like magic until I had to debug YAML files at 2 AM.

\textbf{The Quality Leadership:} Led cross-team quality initiatives. Introduced comprehensive testing, reduced test execution time by 70\%, optimized memory usage.

\textbf{The Skepticism Born:} Developed healthy skepticism of complex orchestration tools. Sometimes the simple solution is actually better.

\newpage

% Page 3: Earlier Career and Personal Projects
\section{The Foundation Years: Learning to Fail Properly}

\subsection{Early Freelancing (2014-2016): The Hustle Phase}
\textbf{The Strategy:} Take every project that comes along. How hard could consulting be?

\textbf{The Reality:} Jumping between PHP, Java, Jenkins, AWS, and Docker projects. Each client had different standards, different problems, different ways of making simple things complicated.

\textbf{The Clients:} YOOCHOOSE (Spring/AWS), Xsite (REST APIs), freshcells (Jenkins/Docker), fotocommunity (Bamboo CI), Silver Tours (infrastructure), AffiliCon (Jenkins), dailypresent (PHP/AWS/Facebook integration).

\textbf{The Lessons:}
\begin{itemize}
\item Client communication is as important as code quality
\item Documentation prevents 3 AM support calls
\item Always scope projects more carefully than you think necessary
\item Some clients become friends, others become learning experiences
\end{itemize}

\textbf{The Pattern:} Most projects involved introducing CI/CD, improving code quality, or migrating to cloud platforms. Apparently, I became ``the Jenkins guy'' without realizing it.

\subsection{The werkenntwen Years (2007-2014): Where It All Began}
\textbf{The Context:} One of Germany's largest social networks, 9+ million users. Started as PHP developer, became Head of Development, returned to backend development.

\textbf{The Management Experiment (2008-2010):} Promoted to Head of Development after 1 year. Company thought: ``He's technical, management can't be that different.''

\textbf{The Management Failures:}
\begin{itemize}
\item Tried to apply software metrics to human performance
\item Focused on technical solutions for people problems
\item Underestimated the complexity of team dynamics
\item Assumed authority would come with the title
\end{itemize}

\textbf{The Technical Contributions:}
\begin{itemize}
\item Migrated from procedural to object-oriented architecture
\item Implemented Zend Framework across major platform components
\item Built REST API for mobile platform
\item Introduced CI/CD with Jenkins and comprehensive testing
\item Scaled architecture for 9+ million users
\end{itemize}

\textbf{The Technology Mistake:} Chose Sphinx over ElasticSearch because ``ElasticSearch looked too exotic.'' The platform later migrated to ElasticSearch anyway.

\textbf{The Redemption:} Returned as Backend Developer (2010-2014) after management stint. Applied lessons learned to build better technical solutions without the pressure of people management.

\textbf{The Company Ending:} werkenntwen was acquired by RTL interactive in 2009 but ceased operations in 2014. Learned that even successful platforms can disappear due to market changes.

\section{Personal Projects: The Experimental Phase}

\subsection{SchulteDev Portfolio Repository (2025)}
\textbf{The Problem:} Maintaining professional portfolio across multiple formats (CV, website, presentations) was becoming a nightmare.

\textbf{The Solution:} Built monorepo with AI-powered CV generation using Claude 4, automated GitHub Actions workflows, and LaTeX document generation.

\textbf{The Irony:} Used AI to generate a document about human professional failures. The future is weird.

\textbf{Technologies:} Node.js, Jekyll, LaTeX, GitHub Actions, Anthropic Claude 4 API

\subsection{AI Document Processing for Tax Software (2025)}
\textbf{The Personal Pain:} Manually entering tax documents is tedious. Surely AI can solve this?

\textbf{The Solution:} Built CLI application using Azure Document Intelligence to automatically extract data from tax documents and integrate with SteuerSparErklärung.

\textbf{The Reality Check:} AI document processing is surprisingly good for structured documents, surprisingly bad for handwritten receipts from small businesses.

\textbf{Technologies:} Golang, Azure Document Intelligence, SQLite

\subsection{Jenkins ↔ GitHub Integration PoC (2018)}
\textbf{The Idea:} Automatic Jenkins setup for GitHub repositories. Because manual CI/CD configuration is for peasants.

\textbf{The Implementation:} Multi-tenant Jenkins architecture with automated webhook configuration on Google Cloud.

\textbf{The Outcome:} Proof of concept worked great. Never turned it into a product because I got distracted by other projects.

\textbf{The Lesson:} Finishing projects is harder than starting them.

\subsection{Crypto-Arbitrage Trading Bot (2015)}
\textbf{The Get-Rich-Quick Scheme:} Build automated trading bot for Bitcoin arbitrage. What could go wrong?

\textbf{The Technical Challenge:} Implemented machine learning algorithms (Weka, LibSVM) for price prediction and automated trading across exchanges.

\textbf{The Reality Check:} By the time the bot was ready, arbitrage opportunities had disappeared. The market adapted faster than my development timeline.

\textbf{The Valuable Lesson:} Sometimes the best business opportunity is the one you don't pursue. Also, cryptocurrency markets are more efficient than they appear.

\textbf{Technologies:} Java, xChange API, Weka, LibSVM, MySQL, JUnit

\newpage

% Page 4: Education, Skills, Career Breaks, Footer
\section{Education: The Long Road to a Degree}

\subsection{Diploma in Computer Science (2004-2012)}
\textbf{University:} University of Koblenz-Landau, Campus Koblenz

\textbf{Degree:} Diploma in Computer Science (equivalent to Master's)

\textbf{Grade:} Distinction (magna cum laude equivalent)

\textbf{Duration:} 8 years (standard: 5 years)

\textbf{The Honest Explanation:} Worked part-time and sometimes full-time as a software developer starting in 2007. Chose practical experience over fast graduation.

\textbf{The Justification:} Learning real-world software development while studying theoretical computer science. By graduation, I had 5 years of professional experience.

\textbf{The Result:} Graduated with distinction despite extended timeline. Turns out practical experience helps with academic understanding.

\textbf{Focus Areas:} Computational Visualistics, Software Engineering, Web Technologies, Human-Computer Interaction

\subsection{Abitur (2003)}
\textbf{School:} Gymnasium Haren (Ems), Germany

\textbf{Advanced Courses:} Mathematics and Physics

\textbf{First IT Contact:} Taught myself PHP using a reference book during high school. Built forums and websites for friends. Discovered computer games, which influenced my decision to study Computer Visualistics.

\textbf{The Irony:} Chose ``Computer Visualistics'' because of games, ended up in enterprise software development. Life rarely goes according to plan.

\section{Skills: What I Actually Know (And What I Pretend to Know)}

\subsection{Programming Languages}
\begin{itemize}
\item \textbf{Java (18 years):} My programming mother tongue. Can write Spring Boot applications in my sleep.
\item \textbf{Golang (4 years):} Surprisingly pleasant to work with. Clean, fast, and doesn't make me want to throw my laptop.
\item \textbf{TypeScript/JavaScript:} Functional knowledge for Node.js and frontend integration. Still occasionally confused by ``this'' binding.
\item \textbf{PHP (8 years):} Foundation language, but I've moved on. Don't judge me for my past.
\end{itemize}

\subsection{Architecture \& Design}
\begin{itemize}
\item \textbf{Self-contained Systems:} Preferred over microservices for business applications
\item \textbf{Event-driven Architecture:} Powerful but requires discipline
\item \textbf{Cloud-native Design:} AWS/Azure/GCP, but prefer managed services over Kubernetes
\item \textbf{Legacy Modernization:} Specialized in incremental migration strategies
\end{itemize}

\subsection{Technologies I Actually Recommend}
\begin{itemize}
\item \textbf{Cloud Platforms:} AWS/Azure managed services, Docker (without Kubernetes when possible)
\item \textbf{Testing:} Contract testing (Pact), comprehensive test automation, JUnit/Jest
\item \textbf{CI/CD:} GitHub Actions, Jenkins, Terraform, automated everything
\item \textbf{Databases:} PostgreSQL, MySQL, Redis (when appropriate)
\item \textbf{Monitoring:} Application Insights, ELK Stack, Grafana
\end{itemize}

\subsection{Technologies I Avoid (And Why)}
\begin{itemize}
\item \textbf{Kubernetes:} Overly complex for most business applications
\item \textbf{Microservices:} Without proper tooling and discipline, worse than monoliths
\item \textbf{Bleeding-edge frameworks:} Prefer stable, proven technologies
\item \textbf{Hype-driven development:} Choose tools for problems, not for resume appeal
\end{itemize}

\section{Career Breaks: The Honest Timeline}

\subsection{Parental Leave (April 2020 -- September 2021)}
\textbf{The Official Reason:} COVID-19 pandemic, combined with family time.

\textbf{The Honest Reason:} Pandemic lockdowns coincided with industry-wide project slowdowns. Freelance opportunities dried up, so I embraced family time.

\textbf{The Unexpected Benefit:} Returned to the market refreshed and with clearer priorities. Sometimes forced breaks are exactly what you need.

\subsection{Personal Travel Break (August -- December 2015)}
\textbf{The Reason:} Wanted to explore different countries and cultures.

\textbf{The Reality:} Sometimes you need to step away from code to remember why you started coding in the first place.

\textbf{The Outcome:} Returned with renewed enthusiasm and different perspectives on problem-solving.

\subsection{Military Service (August 2003 -- March 2004)}
\textbf{The Context:} Mandatory military service in Germany (now abolished).

\textbf{The Assignment:} Air force, basic training in Netherlands, then transferred to Hopsten airbase.

\textbf{The Lesson:} Sometimes you have to do things you don't want to do. At least it wasn't customer service.

\section{Certifications \& Recognition}

\textbf{StackOverflow:} 4000+ reputation points from helping other developers debug their problems

\textbf{Java SE 8 TechCheck:} 93/100 score, 88th percentile (probably expired by now)

\textbf{GitHub:} Multiple open-source contributions, though most are private client repositories

\textbf{Letters of Recommendation:} Multiple positive references from clients, though some companies no longer exist

\vspace{2em}

\hrule

\vspace{0.5em}

\textbf{Disclaimer:} This document was generated using AI assistance (Claude 4) from comprehensive career data. The failures and lessons are real, the self-deprecating humor is intentional, and the technical achievements are factual. If you're still reading this, you might be exactly the kind of person I'd like to work with.

\textbf{Contact:} \href{mailto:mail@schulte-development.de}{mail@schulte-development.de} $\cdot$ \href{https://schulte-development.de}{schulte-development.de}

\end{document}