\documentclass[10pt,a4paper]{article}
\usepackage[margin=0.5in]{geometry}
\usepackage{enumitem}
\usepackage{titlesec}
\usepackage{xcolor}
\usepackage{hyperref}
\usepackage{multicol}
\usepackage{tabularx}

\setlength{\parindent}{0pt}
\setlength{\parskip}{2pt}
\setlength{\columnsep}{15pt}

\titleformat{\section}{\large\bfseries}{\thesection}{1em}{}
\titleformat{\subsection}{\normalsize\bfseries}{\thesubsection}{1em}{}
\titlespacing*{\section}{0pt}{8pt}{4pt}
\titlespacing*{\subsection}{0pt}{6pt}{2pt}

\hypersetup{
    colorlinks=true,
    linkcolor=blue,
    urlcolor=blue,
    citecolor=blue
}

\begin{document}

\begin{center}
{\LARGE \textbf{Markus Schulte}}\\
\vspace{2pt}
{\large \textbf{Anti-CV: Professional Disasters \& Lessons Learned}}\\
\vspace{4pt}
\textbf{Cloud Architect \& Software Engineer}\\
Cologne, Germany\\
\href{mailto:mail@schulte-development.de}{mail@schulte-development.de} | +49 178 7217768\\
\href{https://schulte-development.de}{schulte-development.de} | \href{https://github.com/SchulteDev}{GitHub} | \href{https://linkedin.com/in/markus-schulte}{LinkedIn}
\end{center}

\vspace{6pt}

\section{Professional Disasters Summary}

\textbf{18+ years of surviving software development} through spectacular failures, questionable decisions, and accidentally successful projects. Specialized in turning simple requirements into complex architectures, over-engineering solutions, and learning the hard way that "it works on my machine" isn't a deployment strategy.

\textbf{Core Anti-Expertise:}
\begin{itemize}[leftmargin=10pt,itemsep=1pt]
\item \textbf{Architecture Overkill:} Consistently chose complex solutions when simple ones existed
\item \textbf{Technology Addiction:} Collected bleeding-edge frameworks like Pokemon cards
\item \textbf{Microservices Skeptic:} Learned that distributed systems distribute problems too
\item \textbf{Leadership Learning:} Discovered management is about humans, not just code
\item \textbf{Perfectionism Recovery:} Slowly learning that "good enough" is actually good enough
\end{itemize}

\textbf{Key Disasters That Somehow Worked:}
\begin{itemize}[leftmargin=10pt,itemsep=1pt]
\item Built platforms serving 9M+ users despite choosing every wrong technology first
\item Reduced deployment times from days to minutes after initially making them take weeks
\item Achieved 99.9\% uptime only after experiencing spectacular downtimes
\item Led teams of 5-9 developers after learning leadership isn't about being the smartest person in the room
\end{itemize}

\textbf{My Realistic Slogan:} "Eventually crafting solutions that work despite my best efforts to overcomplicate them"

\section{Recent Professional Failures}

\subsection{November 2024 — Present: Cloud Architect at LBBW}
\textbf{Project:} "Phoenix" IT Modernization for Germany's Major State Bank

\textbf{The Challenge:} Hundreds of legacy applications need Azure migration assessment. Sounds simple, right?

\textbf{My Contribution to Chaos:}
\begin{itemize}[leftmargin=10pt,itemsep=1pt]
\item \textbf{Platform Evaluation Overkill:} Spent weeks evaluating Elastisys Welkin when the existing Azure solution was probably fine
\item \textbf{Architecture Analysis Paralysis:} Created detailed migration assessments for applications that might never migrate
\item \textbf{Template Obsession:} Built "comprehensive frameworks" when simple checklists would have sufficed
\item \textbf{Banking Compliance Reality Check:} Discovered that "move fast and break things" doesn't apply to financial institutions
\end{itemize}

\textbf{Lessons Learned:}
\begin{itemize}[leftmargin=10pt,itemsep=1pt]
\item Enterprise architecture is 20\% technology, 80\% politics and process
\item "Cloud-native" doesn't mean "ignore regulations"
\item Sometimes the best architecture decision is accepting the status quo
\item Banking IT moves at the speed of compliance, not innovation
\end{itemize}

\subsection{October 2021 — December 2024: Cloud Architect at Union Investment}
\textbf{Project:} B2B Platform Modernization (3+ years of "learning experiences")

\textbf{The Grand Plan:} Transform legacy Liferay monolith into modern Self-contained Systems architecture. What could go wrong?

\textbf{My Spectacular Overengineering:}
\begin{itemize}[leftmargin=10pt,itemsep=1pt]
\item \textbf{Microfrontend Madness:} Built custom toolkit when existing solutions existed. Spent 6 months on framework that could have been solved with iframes
\item \textbf{Event-Driven Everything:} Designed event streams for search across 8 domains. Turns out, sometimes a REST API is simpler
\item \textbf{Technology Collecting:} Azure Static Web Apps, Functions, Event Hubs, Web Components, Nx, Lerna, pnpm - collected technologies like infinity stones
\item \textbf{Testing Obsession:} 1000+ tests with 10-minute build times. Proud of parallelization that solved a problem I created
\item \textbf{Architecture Perfectionism:} Spent months on "proper" SCS boundaries when business just wanted working software
\end{itemize}

\textbf{360° Feedback Reality Check:}
\begin{itemize}[leftmargin=10pt,itemsep=1pt]
\item \textbf{Communication Gap:} "Consider pace of technical explanations for less technical stakeholders" - apparently, not everyone wants to hear about event sourcing patterns
\item \textbf{Complexity Addiction:} Team feedback: "Sometimes explanations are too detailed" - guilty of loving my own voice
\item \textbf{Technical Perfectionism:} "Stands up for the right things" - sometimes the right thing is shipping imperfect code
\end{itemize}

\textbf{Accidental Successes:}
\begin{itemize}[leftmargin=10pt,itemsep=1pt]
\item Despite overengineering, achieved 99.9\% uptime and 30-minute deployments
\item Custom microfrontend toolkit actually worked and was adopted company-wide
\item Search solution performed well, even if architecture was unnecessarily complex
\item Team eventually appreciated the technical depth, even if initial explanations were overwhelming
\end{itemize}

\subsection{December 2021 — November 2024: Middleware Engineer at Saloodo!}
\textbf{Challenge:} Sync 10M records between Saloodo and Salesforce. Simple data transfer, right?

\textbf{My Golang Adventure:}
\begin{itemize}[leftmargin=10pt,itemsep=1pt]
\item \textbf{Language Learning in Production:} Chose this project to learn Golang professionally. Testing new languages in production systems - what could go wrong?
\item \textbf{Solo Development Hubris:} Took on critical middleware as sole developer. No team, no backup, no safety net
\item \textbf{Database Schema Denial:} Saloodo's schema was "inconsistent" (malformed JSON, wrong types). Spent weeks building robust error handling for fundamentally broken data
\item \textbf{Perfectionism vs. Reality:} Achieved 70\% test coverage and 5-second test runs. Probably over-tested a simple data sync tool
\end{itemize}

\textbf{Lessons Learned:}
\begin{itemize}[leftmargin=10pt,itemsep=1pt]
\item Learning new languages in production is stressful but effective
\item Sometimes you have to work with terrible data schemas and smile
\item Being the only developer is both terrifying and liberating
\item Golang is great for this kind of work, but Java would have been safer
\end{itemize}

\section{Historical Failures \& Learning Experiences}

\subsection{2018-2020: AWS \& Microservices Reality Check}

\textbf{Trusted Shops (2018-2019):} First encounter with "real" microservices architecture

\textbf{My Naive Expectations:}
\begin{itemize}[leftmargin=10pt,itemsep=1pt]
\item \textbf{Microservices Hype Victim:} Believed the conference talks about independent deployments and team autonomy
\item \textbf{Complexity Shock:} Simple changes required modifications across 5+ services
\item \textbf{Testing Nightmare:} Integration testing became a distributed systems problem
\item \textbf{Monitoring Overload:} Went from monitoring 1 application to monitoring 15+ services
\end{itemize}

\textbf{Reality Check:}
\begin{itemize}[leftmargin=10pt,itemsep=1pt]
\item Microservices distribute problems, not just processing
\item Conway's Law is real - your architecture reflects your org chart
\item The network is not reliable, and neither are your assumptions
\item Sometimes a well-structured monolith is better than poorly designed microservices
\end{itemize}

\textbf{Toom Baumarkt (2016-2017):} Kubernetes Early Adopter Trauma

\textbf{My Container Enthusiasm:}
\begin{itemize}[leftmargin=10pt,itemsep=1pt]
\item \textbf{Kubernetes Complexity:} Chose Kubernetes when Docker Compose would have sufficed
\item \textbf{Operations Nightmare:} Neither ops team nor dev team understood the complexity
\item \textbf{YAML Engineering:} Spent more time writing configs than application code
\item \textbf{Troubleshooting Hell:} "It works locally" became "it works in my cluster"
\end{itemize}

\textbf{Lesson Learned:} Kubernetes is for platform teams, not application teams

\subsection{2007-2014: Leadership Learning at wer-kennt-wen}

\textbf{The Promotion:} From developer to Head of Development at 23. What could go wrong?

\textbf{My Management Disasters:}
\begin{itemize}[leftmargin=10pt,itemsep=1pt]
\item \textbf{Technical Leadership Fallacy:} Assumed being the best developer meant being the best manager
\item \textbf{Human Factor Blindness:} Focused on technical solutions to people problems
\item \textbf{Over-Engineering Management:} Created complex processes when simple conversations would have worked
\item \textbf{Perfectionism Paralysis:} Demanded technical perfection while ignoring team dynamics
\end{itemize}

\textbf{Technology Decisions That Aged Poorly:}
\begin{itemize}[leftmargin=10pt,itemsep=1pt]
\item \textbf{Sphinx vs. ElasticSearch:} Chose familiar over innovative. Platform later migrated to ElasticSearch, proving I should have been braver
\item \textbf{Conservative Technology Choices:} Consistently chose "safe" options that became technical debt
\end{itemize}

\textbf{Lessons Learned:}
\begin{itemize}[leftmargin=10pt,itemsep=1pt]
\item Leadership is about people, not just technology
\item Sometimes the "risky" technology choice is actually safer long-term
\item Management and technical skills are different skill sets
\item Laissez-faire leadership worked better for my personality and team needs
\end{itemize}

\subsection{2016-2024: Atlassian Marketplace Adventures}

\textbf{The Business:} Created commercial Bamboo Server add-ons for code quality integration

\textbf{My Product Development Naivety:}
\begin{itemize}[leftmargin=10pt,itemsep=1pt]
\item \textbf{Niche Market Obsession:} Built for Bamboo Server while the world moved to cloud CI/CD
\item \textbf{Feature Creep:} Added complexity when customers wanted simplicity
\item \textbf{Documentation Procrastination:} Spent more time coding than explaining
\item \textbf{Marketing Allergies:} Excellent at building, terrible at selling
\end{itemize}

\textbf{Accidental Success:}
\begin{itemize}[leftmargin=10pt,itemsep=1pt]
\item 83 customers across 15+ countries despite minimal marketing
\item 7-year sustainable business without really trying
\item First-mover advantage in a niche that actually needed the solution
\end{itemize}

\section{Education \& Skill Development Failures}

\subsection{Academic Procrastination}
\textbf{University of Koblenz-Landau (2004-2012):} Diploma in Computer Science

\textbf{My Time Management Disasters:}
\begin{itemize}[leftmargin=10pt,itemsep=1pt]
\item \textbf{Extended Duration:} Took 8 years for a 5-year degree while working full-time
\item \textbf{Priority Confusion:} Chose immediate income over timely graduation
\item \textbf{Perfectionism Trap:} Spent too much time on perfect solutions instead of good enough
\item \textbf{Work-Study Balance:} Learned the hard way that doing both well is nearly impossible
\end{itemize}

\textbf{Unexpected Benefit:} Real-world experience during studies provided practical context for theoretical knowledge

\subsection{Technology Addiction Timeline}
\textbf{My Pattern of Collecting Shiny Technologies:}
\begin{itemize}[leftmargin=10pt,itemsep=1pt]
\item \textbf{2007-2014:} PHP, MySQL, Apache - The basics (when they were actually enough)
\item \textbf{2014-2018:} AWS, Docker, Kubernetes - The cloud hype train
\item \textbf{2018-2021:} Microservices, Event Sourcing, Contract Testing - The architecture astronaut phase
\item \textbf{2021-2024:} Microfrontends, Web Components, Event Streams - The frontend complexity explosion
\end{itemize}

\textbf{Pattern Recognition:} Consistently chose complex solutions for simple problems

\subsection{Career Breaks as Learning Experiences}

\textbf{April 2020 — September 2021: Pandemic Parental Leave}
\begin{itemize}[leftmargin=10pt,itemsep=1pt]
\item \textbf{Timing Failure:} Took break during industry slowdown, missing recovery opportunities
\item \textbf{Skill Atrophy:} Discovered how quickly technical skills fade without practice
\item \textbf{Market Disconnect:} Returned to find the job market had evolved significantly
\end{itemize}

\textbf{Lessons Learned:} Family time is valuable, but professional re-entry requires strategic planning

\subsection{Current Skill Gaps \& Ongoing Failures}
\textbf{Technologies I Still Don't Understand:}
\begin{itemize}[leftmargin=10pt,itemsep=1pt]
\item \textbf{Machine Learning:} Built crypto trading bot in 2015, still don't understand the math
\item \textbf{Modern Frontend:} React, Angular, Vue - chose Web Components instead
\item \textbf{DevOps Tools:} Know enough to be dangerous, not enough to be safe
\item \textbf{Database Administration:} Can use them, can't optimize them
\end{itemize}

\textbf{Soft Skills Development Areas:}
\begin{itemize}[leftmargin=10pt,itemsep=1pt]
\item \textbf{Marketing:} Technical excellence doesn't sell itself
\item \textbf{Business Development:} Great at building, terrible at positioning
\item \textbf{Communication:} Still explaining technical concepts to non-technical stakeholders
\item \textbf{Time Management:} Perpetually optimistic about estimates
\end{itemize}

\section{Conclusion: Lessons from Professional Disasters}

After 18+ years of making mistakes professionally, I've learned that:

\begin{itemize}[leftmargin=10pt,itemsep=1pt]
\item The best architecture is the one your team can understand and maintain
\item Simple solutions scale better than complex ones
\item Technology choices matter less than team dynamics
\item Failure is the best teacher, but expensive
\item Sometimes "good enough" is actually perfect
\item Experience is what you get right after you needed it
\end{itemize}

\textbf{Current Status:} Still learning, still failing, still improving. Available for new disasters and the lessons they provide.

\vspace{10pt}
\hrule
\vspace{4pt}
\footnotesize
\textbf{Anti-CV Disclaimer:} This document was generated using AI assistance to transform a traditional CV into a humorous but professional reflection on career failures and lessons learned. While the tone is self-deprecating, the underlying experiences and technical competencies are accurate. The author remains available for serious professional engagements despite this comedic presentation.

\end{document}